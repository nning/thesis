\chapter{Guides}
\label{cha:guides}

\section{Basic Installation and Usage}

	It is assumed that a recent version of Ruby is installed. At the time of
	writing, Ruby 2.2.0 is the latest stable version released. Documentation
	about how to install Ruby on different \acp{OS} can be obtained from the
	official Ruby
	website\footnote{\url{https://www.ruby-lang.org/en/documentation/installation/}}.

	In case a new \ac{Rails} application is to be used, it is created with the
	following command.

	\begin{lstlisting}[gobble=4]
		rails new example
	\end{lstlisting}

	If a \ac{Rails} application is existent, David can be included as a
	dependency by including the following line into \texttt{Gemfile}.

	\begin{lstlisting}[gobble=4]
		gem 'david'
	\end{lstlisting}

	David will automatically hook into Rails to make itself the default Rack
	handler so that \texttt{rails s} starts David by default. If the
	application has to be used via \ac{HTTP}, WEBrick can be started by
	executing \texttt{rails s webrick}. After the server is started, the
	\ac{Rails} application is available at \texttt{coap://[::1]:3000/} by
	default.

	\href{https://addons.mozilla.org/de/firefox/addon/copper-270430/}{Copper}
	is a \ac{CoAP} client for Firefox and can be used for development. The
	\href{https://github.com/nning/coap}{Ruby coap gem} is used by David for
	example for message parsing and also includes a command line utility (named
	\texttt{coap}) that can also be used for development.

	As \ac{CoAP} is a protocol for constrained environments and machine to
	machine communications, returning \ac{HTML} from controllers will not be of
	much use. \ac{JSON} for example is more suitable in that context.  David
	works well with the default ways to handle \ac{JSON} responses from
	controllers such as \texttt{render json:}. You can also utilize Jbuilder
	templates for easy generation of more complex \ac{JSON} structures.

	\ac{CBOR} \cite{cbor} can be used to compress \ac{JSON} documents.
	Automatic transcoding between \ac{JSON} and \ac{CBOR} is activated by
	setting the Rack environment option \texttt{CBOR} or
	\texttt{config.coap.cbor} in your Rails application config to
	\texttt{true}.

\section{Running the \acf{RD}}

	Clone the repository.

	\begin{lstlisting}[gobble=4]
		git clone https://github.com/nning/core-rd.git
		cd core-rd
	\end{lstlisting}

	Install dependencies and run database migrations.

	\begin{lstlisting}[gobble=4]
		gem install bundler
		bundle --deployment --without 'development doc test'

		export RAILS_ENV=production
		rake db:migrate
		rake db:seed
	\end{lstlisting}

	Generate a secret token for the production environment. It has to be pasted
	in the production section of \texttt{config/secrets.yml}.

	\begin{lstlisting}[gobble=4]
		rake secret
	\end{lstlisting}

	Start the server.

	\begin{lstlisting}[gobble=4]
		rackup
	\end{lstlisting}
