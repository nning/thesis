\chapter{Objectives}
\label{cha:objectives}

Corresponding to the primary research question stated in
\autoref{cha:intro:questions}, the higher objective of our work is to determine
the possibility of using Rack based web frameworks originally designed for
\ac{HTTP} in the \ac{IoT} utilizing \ac{CoAP}. We utilize \acf{TDD} for the
software components. There are also some general design goals to the code such
as code quality, modularity, and sustainability through readability and a high
test coverage for example.

\section{\acs{CoAP} Server}

	\subsection{Protocol Implementation}

		As already shown in \autoref{cha:background:coap:implementations},
		there are numerous different implementations of the \acf{CoAP} in many
		programming languages and with differing design goals. We will examine
		some of them to determine the implementation to base the server
		developed during this work on. The support for a recent standard
		version \cite{coap} and extensions for observing resources
		\cite{observe} and block-wise transfers \cite{block} is particularly
		important. We will analyze and design different aspects of the protocol
		such as Message Correlation and Deduplication as well as different
		response types from a server's perspective and implement them
		accordingly. General design goals are the robustness and
		interoperability of the \ac{CoAP} implementation.

	\subsection{Concurrency and Performance}
	\label{cha:objectives:concurrency}

		For the handling of frequent sensor data updates from a huge number of
		\ac{IoT} devices, the concurrency model has to be chosen fittingly.
		Concurrent handling of requests is therefore one design goal of the
		server implementation. We will analyze which models are viable for
		applications in the \ac{IoT} and how possible models integrate with
		\ac{Rails}. Neither the server itself, nor the \ac{CoAP} implementation
		or the application framework may block the concurrent handling of
		further requests. For a realistic objective for the concurrency level
		and the maximum number of requests per second of the implementation,
		benchmarks of other implementations should provide for an overview. It
		is important to consider the comparability of different benchmarks, as
		they differ at least on two levels: First, the hardware configurations
		range from ultra-portable, energy-saving developer notebooks, to
		several full-fledged, high-end servers. Secondly, the choice of
		frameworks for testing is important. A benchmark testing a plain Hello
		World Rack application is not comparable with its \ac{Rails} or
		\emph{Sinatra} counterpart, because of the framework overhead. For this
		work, only the server performance has to be measured, so benchmarking
		different web servers on several Ruby VMs serving a plain Hello World
		Rack application would be of most value. The server implemented during
		this work does not have to scale as well as \emph{Californium} for
		example, which according to a paper among others written by
		\emph{Californiums} original author Matthias Kovatsch handles almost
		400,000 requests per second \cite{cf-scale} via \ac{CoAP} on server
		hardware. For Ruby applications, those values are reachable over
		\ac{HTTP} with server stacks not written in Ruby such as TorqBox on
		JRuby or nginx proxying Unicorn on \ac{MRI} for
		example\footnote{\urlRubyWebBenchOne}. With \ac{HTTP} web servers
		written in Ruby running plain Rack applications (no further frameworks
		like \ac{Rails}) scale up until approximately 7.600 requests per
		second\footnote{\urlRubyWebBenchTwo}, so it is expected that with
		\ac{CoAP} this number could be improved in comparison to \ac{HTTP}. For
		a server scaling this well still much optimization work has to be done,
		which is not the focus of this work. The more conservative number of
		5.000 requests per second shall be aimed at, for now. Implicated
		objectives by these performance goals are the portability of the server
		and the \ac{CoAP} implementation to other Ruby \acp{VM} than \ac{MRI},
		such as JRuby for example, and the choice of a server concurrency model
		which scales well with the provided hardware.

	\subsection{Rack Interface}

		As many other Ruby web frameworks, \ac{Rails} interacts via the Rack
		interface with the web server. The main targeted web application
		framework will be \ac{Rails} but the server implementation shall also
		be tested against different other Rack middlewares and application
		frameworks. It shall be clarified in which extent this interface
		suffices for the characteristics of the \acf{CoAP}. The translation
		between the Rack interface based on \ac{HTTP} and \ac{CoAP} has to
		function seamlessly and be as complete as possible.

	\subsection{\acs{DTLS}}

		The server has to be designed to allow the integration of \ac{DTLS}
		\cite{dtls} as a means of transport encryption. We do not pursue an
		actual implementation. The implications of \ac{DTLS} in the context of
		this work shall be examined. Functionality such as observe or certain
		Rack middleware might possibly be derogated.

\section{Protocol Translations}

	% TODO	Intro

	\subsection{Headers and Payload Formats}

		% TODO	Lengthen JSON/CBOR

		The software implemented during this work shall enable an integrated
		development of applications for the traditional, human-readable web and
		machine-to-machine communication in the \ac{IoT}. It will be clarified
		how the \ac{HTTP} and \ac{CoAP} Content Negotiation protocols (see
		section 5.3 of \cite{http-1} and 5.5.4 of \cite{coap}) can be utilized
		for the purpose of separating those client groups from a developers
		perspective in \ac{Rails}. Other functionality of \ac{HTTP} based upon
		headers applicable to \ac{CoAP} shall also be preserved using an
		appropriate translation of the headers (for example the \texttt{ETag}
		header which is defined in section 2.3 of \cite{http-2}). Also the
		implications of a transparent content type mapping for responses by the
		web framework between a data serialization format of the traditional
		web and more resource-saving ones for the \ac{IoT} will be emphasized.
		A transcoding of resource representation formats between \ac{JSON}
		\cite{json} and \ac{CBOR} \cite{cbor} will be implemented.

	\subsection{Resource Discovery and Observe}

		Furthermore, special properties of \ac{CoAP} such as the Resource
		Discovery (see section 7.2 of \cite{coap}) and Observe \cite{observe}
		will be considered and integrated appropriately. The Resource Discovery
		shall be integrated into the web framework (for example \ac{Rails})
		without any developer interaction for resources for which a interaction
		with \ac{CoAP} is reasonable. The functionality of an \ac{RD} \cite{rd}
		shall be facilitated by pre-implemented, configurable, and well-tested
		modules. It will also serve as application to demonstrate the
		implementation properties. In order to support Resource Discovery and
		the implementation of an \ac{RD}, Group Communication for the \ac{CoAP}
		\cite{coap-group} will be integrated. Additionally, a way to integrate
		the asynchronous nature of Observe and the rather synchronous ones of
		Rack and \ac{Rails} will be designed and implemented. 

\section{Evaluation}

	For the evaluation, it shall be analyzed to which extent the specified
	goals of functionality are met, how the performance of the implementation
	compares to related work and how \ac{Rails} developers respond to given
	tasks in connection with the developed software modules.

	Unit tests emerge during the \acf{TDD} that cover most parts of the
	implementation source code and that specify and inspect the functionality
	of individual modules, classes and methods of the different components. The
	coverage of source code by tests (\emph{Code Coverage}) will be measured.
	For the demonstration application, unit, functional and integration tests
	shall be created that besides testing the application itself also
	demonstrate \ac{TDD} on the covered software stack with methods common in
	\ac{Rails}. The \ac{CoAP} library will be tested for its interoperability
	with other implementations in an automated way.

	The \ac{CoAP} server will be compared to other implementations by
	performance tests. In doing so we will also measure the consumption of
	resources. We will also compare the performance of different Rack based
	frameworks in connection with the server. For the applicability of the
	complete stack for the usage of control nodes in home automation networks,
	a test installation will be made on home router hardware. The data gained
	by the performance tests will be used to reflect on design decisions as the
	kind of solution for the realization of the \ac{CoAP} server.

	To evaluate the developer friendlyness and the seamlessness of the
	integration in \ac{Rails}, experienced Rails developers are going to be
	interviewed after they solved a given task. This task shall contain
	\ac{CoAP} specifics that the developers do not necessarily know of.
