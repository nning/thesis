\chapter{Related Work}
\label{cha:related-work}

In this chapter we examine which solutions exist that partly answer our
research questions and give an introduction to possible sources in that
contexts.

\section{Frameworks}

	Generally, \ac{CoAP} server implementations provide their own \ac{API} for
	applications in constrained environments to be crafted and do not support
	common frameworks (see a Hello World server application for
	\emph{Californium}\footnote{\urlCaliforniumHelloWorld} and a exemplary
	server with several resources based on
	\emph{libcoap}\footnote{\urlLibcoapExample}, for example). Those \acp{API}
	can also be seen as the framework in which applications are developed in.
	Examples are \emph{Californium} \cite{californium}, \emph{Erbium}
	\cite{erbium} and \emph{libcoap} \cite{libcoap}. For applications on
	constrained nodes the absence of full-featured frameworks is not
	remarkable, because \ac{CPU} and memory resources are quite limited. For a
	definition of the constraints see \cite{terminology}. Applications which do
	not have to be especially optimized on limited resources can profit from
	framework utilization.
	
	At the time of writing, we could only find one integration of a \ac{CoAP}
	server with a common web framework. \emph{Web\acs{IoT}} \cite{webiot} can
	be seen as a framework for constrained applications. It supports different
	protocols and is based on the \ac{GWT}. However, as it is focusing on
	graphical applications it ships with a mandatory \ac{HTTP} web interface
	and allows application development only as plugins. It is also not
	optimized for the features of a specific protocol but tries to aggregate
	sensor data protocol independently.

	In the context of Ruby, Rack is the most widely adopted interface between
	web servers and applications and it is supported by many web application
	frameworks (see \autoref{table:background:rack}). The Rack specification
	\cite{rack} lists the Python Web Server Gateway Interface \cite{pep333} as
	a source of inspiration. Rack is chosen for this work because of its spread
	and provision of a solid base.

	There are many \ac{HTTP} web servers written in Ruby that support the Rack
	interface. Although they are probably designed in a substantially different
	way than \ac{CoAP} servers, they can be analyzed regarding to their Rack
	implementation and used for comparison in the performance evaluation of our
	work. Notable examples are \emph{Reel}, \emph{Thin}, and \emph{Unicorn}
	(see \cite{reel}, \cite{thin}, and \cite{unicorn}).

\section{Scalability}

	The inability of web servers with concurrent request handling based on
	processes or threads to scale up to 100.000 requests per second led to the
	framing of \emph{The C10K problem} \cite{c10k}. One possibility to solve
	this problem is event-driven architecture as it is discussed in the Reactor
	pattern \cite{reactor-1} \cite{reactor-2} designed by Douglas C. Schmidt
	and others. Matthias Kovatsch provides a detailed overview of web server
	architectures with regard to scalability in his dissertation thesis
	\citetitle{scalable-iot} \cite{scalable-iot}.

	For Ruby there are a number of libraries targeting the simplification of
	the development of concurrent programs. For this work especially
	\emph{Celluloid} \cite{celluloid} is considered. The basics of concurrency
	and \emph{Celluloid} are also highlighted in
	\autoref{cha:background:concurrency}.

	Especially \emph{Californium} is promising as a source of inspiration for
	performance and concurrency oriented design. According to a paper among
	others written by its original author Matthias Kovatsch, \emph{Californium}
	handles almost 400,000 requests per second \cite{cf-scale}. Some
	observations from the \emph{Celluloid} source code are collected in
	\autoref{cha:background:coap:concurrency}.

\section{\acs{CoAP}}

	Current and comprehensive lists of \ac{CoAP} implementations are to be
	found in the Wikipedia \cite{coap-list-1} and on the website
	\href{http://coap.technology}{coap.technology} \cite{coap-list-2}.
	Especially implementations in C/C++, Java or -- of course -- Ruby with a
	server component are qualified for a Ruby integration and therefore
	suitable to be based upon in the context of this work. To which extent the
	adaption of an already existing library is beneficial and the corresponding
	design questions are discussed in \autoref{cha:background:coap} and
	\autoref{cha:design:server:coap}.

	Independent of their qualification for a Ruby integration, some
	implementations can give valuable input regarding to the architecture and
	design of the basic protocol features such as deduplication (see section
	4.5 of \cite{coap}) or extensions such as block-wise transfers \cite{block}
	or observe \cite{observe}. Although this work focuses on the server side,
	also client implementations are interesting for their way of message
	parsing, usage in testing of our server implementation and their
	transmission layer with regards to code reuse. However, analyzing every
	implementation in detail would not be possible because of the time
	constraints of this thesis.

	Both the \emph{Cross-Protocol Proxying between CoAP and HTTP} section of
	\cite{coap} (see section 10) and the \emph{Guidelines for
	\ac{HTTP}-\ac{CoAP} Mapping Implementations} \cite{coap-mapping} can be
	consulted regarding to translations of \ac{CoAP} headers and protocol
	features to \ac{HTTP} and vice versa. However, simply proxying an ordinary
	Rails application would not suffice. An objective for our solution is that
	it supports a \ac{CoAP} specific development and therefore should be
	explicit about that protocol. The mapping guidelines also make some general
	statements about the transparent conversion of one content format into
	another, called \emph{Transcoding} (see section 6.4 of
	\cite{coap-mapping}).

	We did not find any Open Source client or server implementations supporting
	the \ac{RD} draft \cite{rd}. The only exception is a server implementation
	contained as an example in the \emph{txThings}\footnote{\urlTxThings}
	project. However, it implements an older version of the draft and is
	missing functionality. The \ac{CoAP} article in the Wikipedia
	\cite{coap-list-1} lists \emph{CoAPSharp} as supporting the \ac{RD} draft
	but its code contains no indications about that.

\section{\acs{DTLS}}

	\ac{CoAP} depends on \ac{DTLS} for security (see section 9 of \cite{coap}).
	The \acs{DICE} \ac{WG} of the \ac{IETF} works on supporting the use of
	\ac{DTLS} in constrained environments. The main draft that is currently
	worked on defines a profile of \ac{DTLS} 1.2 \cite{dtls} (and \ac{TLS} 1.2
	\cite{tls}) \enquote{that offers communication security for \ac{IoT}
	applications} \cite{dtls-profile}. Another draft aims to standardize the
	\ac{DTLS} handshake on top of \ac{CoAP} to make use of block-wise transfers
	\cite{codtls}. There is a Ruby
	library\footnote{\url{https://rubygems.org/gems/codtls}} that implements
	this draft in an alpha state. An early draft \cite{dtls-multicast}
	considers securing multicast communications in that context. The
	integration of \ac{DTLS} into \emph{Californium} is discussed in a master's
	thesis from 2012 \cite{californium-dtls}.
