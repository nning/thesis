\chapter{Design}
\label{cha:design}

The main component being developed through this work is a \ac{CoAP} server able
to serve responses from frameworks compatible to the Rack interface. Section
\ref{cha:design:server} discusses interfaces, libraries and the concurrency
model, and \autoref{cha:design:protocol} analyzes translations between the
\ac{HTTP} and \ac{CoAP} protocols.

\section{\acs{CoAP} Server}
\label{cha:design:server}

	The code will be modularized with mixins (Ruby modules included in
	classes), because these need less memory compared to the instantiation of
	classes. Where tuples of data are needed (as keys for the message
	deduplication cache, for example), structs were considered instead of
	dynamically growing types like arrays. However, a performance advantage of
	structs could not be measured (see
	\href{https://github.com/nning/david/blob/master/experiments/structs.rb}{\texttt{experiments/structs.rb}}
	of the david repository). On \ac{MRI}, both Arrays and Structs are
	implemented in C.

	\subsection{Rack Interface}
	\label{cha:design:server:rack}

		The stable Rack version used in development is
		1.6.0\footnote{\urlRack}. As a source of the Rack interface
		specification the version from the current master branch \cite{rack} is
		referenced here, because it was not possible to find a readably
		formated version of the specification for Rack 1.6.0.
		
		From the servers perspective, on the Rack side only the environment has
		to be provided correctly and completely and the application's
		\texttt{call} method has to be invoked with that environment on
		incoming requests. The \texttt{call} invocation returns the described
		\texttt{Array}, which contains values specific to \ac{HTTP} and
		therefore has to be translated to \ac{CoAP}. For details on the
		translation of incoming \ac{CoAP} messages and outgoing Rack responses
		see \textref{cha:design:protocol}. There are Rack specific environment
		options which are used to pass in objects for input, errors, or for
		\emph{hijacking} the socket (see \textref{cha:design:protocol:observe})
		and options for concurrency for example. An exemplary Rack environment
		is given in \autoref{lst:rack:env} as a Ruby Hash. A more detailed
		description of the requirements of the Rack environment is given in its
		specification \cite{rack}.

		\begin{figure}
			\begin{lstlisting}[gobble=8,caption={Exemplary Rack environment},label=lst:rack:env]
				{
					'REMOTE_ADDR'       => '::1',
					'REMOTE_PORT'       => '41414',
					'REQUEST_METHOD'    => 'GET',
					'SCRIPT_NAME'       => '',
					'PATH_INFO'         => '/.well-known/core',
					'QUERY_STRING'      => 'href=test',
					'SERVER_NAME'       => '::',
					'SERVER_PORT'       => '5683',
					'HTTP_ACCEPT'       => 'application/json',
					'rack.version'      => [1, 2],
					'rack.url_scheme'   => 'http',
					'rack.input'        => #<StringIO:0x007f3d48021570>,
					'rack.errors'       => #<IO:<STDERR>>,
					'rack.multithread'  => true,
					'rack.multiprocess' => true,
					'rack.run_once'     => false,
					'rack.logger'       => #<Logger:0x007f3d644e52e8>
				}
			\end{lstlisting}
		\end{figure}

		Choosing the server when running from Rack configuration files
		(\texttt{config.ru}) or when starting the server in a \ac{Rails}
		project for example is possible by providing a Rack handler for the
		\ac{CoAP} server and registering it. After registration, the desired
		handler can be chosen with a parameter to \texttt{rackup} or
		\texttt{rails server}. A Ruby code example of a basic Rack handler
		prototype is shown in \autoref{lst:rack:handler}. The handler from the
		example would be started by \texttt{rackup -s example} or \texttt{rails
		s example}.

		\begin{figure}
			\begin{lstlisting}[gobble=8,caption={Rack handler prototype},label=lst:rack:handler]
				# The Rack library is required.
				require 'rack'

				# Namespace for register method.
				module Rack::Handler
					# The custom handler class.
					class ExampleHandler
						def self.run(app, options = {})
							# Blockingly start actual server..
						end
					end

					# Register handler class as 'example'.
					register(:example, ExampleHandler)
				end
			\end{lstlisting}
		\end{figure}

		The \texttt{.well-known/core} interface (see \cite{link} and section
		7.2 of \cite{coap}) could be implemented as a Rack middleware. A
		default \ac{Rails} application contains already numerous Rack
		middlewares for many possible purposes, some of them not applicable to
		\ac{CoAP} based communications. \texttt{ActionDispatch::Cookies} for
		example manages \ac{HTTP} Cookie handling. For now, we do not provide
		any \ac{CoAP} features that are translated to \ac{HTTP} using Cookies.
		It has to be evaluated if the removal of this middleware actually
		breaks code which might be useful in the context of constrained
		application development. \texttt{ActionDispatch::DebugExceptions}
		renders exceptions as \ac{HTML}, which is way to verbose for
		development with a \ac{CoAP} client. Both these examples and other
		middleware are not needed. Therefore a way has to be found to
		deactivate unessential middleware in a \ac{Rails} \ac{CoAP}
		application. With a component called Railtie \cite{railtie} it is
		possible to run code from a gem upon initialization of a \ac{Rails}
		application and to provide configuration options which can be used from
		inside the applications configuration files. The middleware cleanup can
		be configurable through a Rails option. A middleware catching
		exceptions thrown in a Rack application and converting them to a
		description for the programmer (\ac{JSON} for example) would support
		the development process. For Rails, such a middleware exists with
		\texttt{ActionDispatch::ShowExceptions} but it does not support
		including exception details in the \ac{JSON} body.
		
		A \texttt{Rack::Lint} class instance can be inserted into the
		middleware stack to automatically test the Rack environment for
		conformity to the Rack specification \cite{rack}.

	\subsection{Protocol Implementation}
	\label{cha:design:server:coap}

		% TODO	More on server protocol implementation goals.
		% TODO	More design in relation to goals.
		% TODO	Structure
		% TODO	Vertiefung (e.g. FSM, etc.)

		The Rack interface has to be pure Ruby code in any case, so it is
		suggested to also use a protocol implementation integratable into Ruby.
		Compared to other languages, it is possible to include C/C++ and Java
		code easily and without indirections like socket communications and
		serialization to connect a non-Ruby component with a Ruby one.
		\autoref{table:background:coap:libraries} gives an overview of
		compatible \ac{CoAP} libraries in different languages. So even though
		the server implementation in another language could promise performance
		benefits (see \autoref{cha:design:server:concurrency}), a Ruby solution
		is preferred here. Both C/C++ and Java tie the Ruby \ac{VM} down to a
		solution written in the respective language. Using a \ac{CoAP}
		implementation in C/C++ would require to write a wrapper C extension
		which uses the Ruby \ac{API} to expose the functions to Ruby code. This
		makes the usage in JRuby impossible, since as of JRuby 1.7 the C
		extension support is deprecated\footnote{\urlJRubyCExt}. On the other
		hand, a \ac{CoAP} implementation in Java will not run in any Ruby
		\ac{VM} outside of a Java \ac{VM}.

		A pure Ruby implementation of \ac{CoAP} can be based upon an existing
		message parser written by Carsten Bormann, as it was done in the
		\emph{SAHARA}\footnote{\urlSaharaOne}\textsuperscript{,}\footnote{\urlSaharaTwo}
		and \emph{GOBI}\footnote{\urlGobi} projects of the communication
		networks research group (AG Rechnernetze\footnote{\urlAgrn}) at the
		Universität Bremen. The \emph{GOBI} implementation has been published
		on github.com\footnote{\urlGobiCoap} and features client functionality,
		basic block-wise transfer and observe support. But it requires some
		refactoring work to meet the objectives of this work such as code
		quality and non-existence of side effects for concurrency support. The
		Code Climate report\footnote{\urlGobiCoapCC} on the library lists
		highly complex classes and much code duplication (both indicating a bad
		structure), and many "code smells" (indicating non-conformance to the
		basic commonalities of Ruby style guides \cite{ruby-style} for
		example). As of the current commit
		(\href{https://github.com/SmallLars/coap/commit/1cd124427b57e84543f3ee6ff2d9148e8499393a}{1cd1244})
		in its git repository, the code climate scores 0.7 (with 4.0 being the
		best possible result) and the test coverage as measured by coveralls.io
		amounts to 88.19\%. Arguments for forking this library and developing
		it further under this name are that it is already present on
		rubygems.org as \emph{coap} and that it enables quick prototyping.
		Although we can not be sure, the code developed during this thesis can
		be merged back again into the original project.
		
		As a \ac{CoAP} \textbf{server} implementation is needed in the context
		of this work, one of the implementation goals regarding the fork of
		this library is making it usable in the context of a server.
		Abstraction and modularization to enable code reuse and testability are
		methods to accomplish that. The current client behavior is contained
		in one huge method that is called recursively in some cases. One case,
		for example, is the assembly of single block-wise \ac{CoAP} messages.
		There are recursion limits to hinder huge resource consumption that are
		also preventing more than 10 outgoing and more than 50 incoming blocks.
		Sending, receiving, retransmissions, parsing, reassembly of block-wise
		transfers, and other protocol states or operations can be organized in
		a transmission layer (documented for example as finite state machines
		in the \ac{CoAP} Implementation Guidance draft \cite{coap-guide}) that
		is reusable by the server, thus avoiding code duplication. Same goes
		for other common functionality such as utilities for handling
		block-wise transfer \cite{block}. A more close implementation of the
		\ac{CoAP} protocol is a further implementation goal that has several
		advantages.  By improving matching and generation of message IDs and
		tokens, security and reliability can be enhanced. With avoidance of
		computation on bogus messages, application state both on the server and
		client side can be minimized. The \ac{API} of the client is planned to
		be cleaned up and simplified. For example, the port argument should be
		optional for requests; obviously, the \ac{CoAP} default port can be
		used, if the argument was omitted. As the server aims for portability
		between Ruby \acp{VM}, also the common library and client component
		have to support them. \emph{Celluloid::IO} \cite{celluloid-io}, which
		is described more detailed in the following section, can not only be
		used for the server, but also for the client to realize non-blocking
		socket operations. To further support concurrency, the client and
		common library code will be revised regarding to globally shared state
		and side effects. Tests for the server component are intended to be
		implemented utilizing RSpec \cite{rspec}. New tests for the client
		component will also be written in RSpec and some of the old tests will
		be ported during the development. For the client, test goals consist of
		increasing the test coverage (for example through modularization),
		decoupling of existing tests from remote hosts, and increasing testing
		speed. Only optional integration or protocol implementation
		compatibility tests should remain dependent on remote hosts. For the
		Resource Discovery (see section 7 of \cite{coap}) both on server and
		client side, the \ac{CoRE} Link Format \cite{link} which describes
		(available) \ac{CoAP} resources is necessary to be implemented.

		\subsubsection{Duplicate Detection and Message Correlation}

			The server will send cached response messages on requests
			identified as duplicates (see section 4.5 of \cite{coap}). Old
			entries in the response cache have to be \enquote{garbage
			collected}. The cache used can also be utilized for the correlation
			of answers to sent messages. As the message receiving, parsing and
			dispatching happens in a single threaded event loop, the cache can
			be accessed without locking. However, other threads such as the
			garbage collector for example need to access the cache
			concurrently. This can be accomplished in a thread safe manner by
			using messaging among actors. A simple implementation of this cache
			could utilize a Ruby Hash with endpoint address and message ID as
			keys and the message and a timestamp for retransmissions and
			garbage collection as values. Optimizations are discussed in
			\autoref{cha:design:server:concurrency}. Retransmission of
			\texttt{CON} messages that received no \texttt{ACK} or \texttt{RST}
			answer in a certain time can be handled by another thread operating
			on the mentioned cache. Additionally, the cache has to include a
			timeout and a retransmission count per entry to keep track of
			retransmissions (see section 4.2 of \cite{coap}). However, if the
			server does not send any confirmable messages, it does not have to
			perform retransmissions.

		\subsubsection{Separate Responses}
		\label{cha:design:server:coap:separate}

			If the response to a confirmable request takes longer than the
			retransmission timeout of the client, separate responses can be
			used to prevent a retransmission (see section 5.2.2 of
			\cite{coap}). Support for separate responses requires a possibility
			to run application code in the background and to later send the
			resulting response message. When using Rack with \ac{HTTP}, the
			server has to keep a connection alive and later send an
			asynchronous response. Therefore this response has to be passed to
			the server. Some methods to pass an asynchronous response to the
			server have been developed in the Rack community. \emph{Rack
			hijack} (see \autoref{cha:design:protocol:observe}) is the only one
			being part of the Rack specification \cite{rack}. Providing a
			Proc-like as value to the Rack environment key
			\texttt{async.callback} is another method that was integrated into
			\emph{Thin} \cite{thin}. Through \texttt{throw :async} or a
			special Rack response (\texttt{[-1, \{\}, []]}) it is signalled
			that the response is obtained via the \texttt{async.callback}.
			Another method implemented in \emph{Thin} is a \enquote{deferrable
			body} object\footnote{\urlThinDeferrableBody}. In each of these
			cases, the middleware used must be aware or is otherwise
			circumvented\footnote{\urlRackAsync}. \emph{Rack hijack} suffers
			from the same problem. \emph{Reel} -- interesting because also
			based on \emph{Celluloid} -- does not have a solution to this
			problem that works with the Rack interface.
			
			As \ac{CoAP} is not constrained to keep a connection alive, there
			is another way of implementing separate responses that is also
			mentioned in section 5.2.2 of \cite{coap}. Before initiating the
			generation of a response by the application, the server can start a
			timeout. When the timeout finished before a response is available,
			the server sends an empty \texttt{ACK} message indicating a
			separate response. The actual response generated by the application
			then has to be sent as a confirmable or non-confirmable message. A
			method can be provided with which an application programmer can
			immediately signal a separate response if a long response time is
			foreseeable.

			\autoref{lst:coap:separate} shows an exemplary source code from the
			application programmer's side. The call of the \texttt{separate}
			method would send an immediate response to the client by messaging
			the server actor. To actually asynchronously run the code, the
			block and the controller instance is passed to a separate
			\emph{Celluloid} actor. This single actor or actor from a thread
			pool would then \texttt{call} the block in the context of the
			controller instance. The instance would have to persist until the
			asynchronous code finished running. The controller response would
			have to be translated like it is in a piggybacked case. The timeout
			method mentioned in the previous paragraph enables also the
			separate answer passing the Rack middleware stack but the
			generation of the response is happening synchronously. For the
			\texttt{seperate} method extension it has to also be ensured, the
			response is passing the Rack middleware stack.

			% TODO	Does the separate answer pass the middleware stack?

			\begin{figure}
				\begin{lstlisting}[gobble=10,caption={Seperate Response},label={lst:coap:separate}]
					class ThingsController < ApplicationController
						def show
							separate do
								render json: Thing.find(params[:id])
							end
						end
					end
				\end{lstlisting}
			\end{figure}

	\subsection{Concurrency and Performance}
	\label{cha:design:server:concurrency}

		% TODO	Revise text and add details.
		% TODO	Why single threaded, event loop, non-blocking?
		% TODO	Benchmark prototype measurements.

		To meet the performance objectives of this work (as stated in
		\autoref{cha:objectives:concurrency}), some consideration is necessary.
		As layed down in \autoref{cha:background:concurrency}, we will use
		\emph{Celluloid} \cite{celluloid} as concurrent object
		\enquote{framework} and the \emph{Celluloid::IO} \cite{celluloid-io}
		reactors for receiving from sockets.

		There are different web server architectures. We choose a \ac{SPED}
		architecture \cite{sped} for its reasonable performance and scalability
		as well as its ease of implementation. Some experiments were conducted
		with different message handling models that are documented in
		\href{https://github.com/nning/david/blob/0.4.3/experiments/concurrency/stub.rb}{\texttt{experiments/concurrency/stub.rb}
		inside the David source code repository}. Message handling, parsing,
		and dispatching happens inside an asynchronous event loop run from a
		\emph{Server} actor of which a single instance (thread) is existent.
		\emph{Celluloid::IO} abstracts the I/O selection independent from the
		\ac{OS} using \emph{nio4r}\footnote{\urlNio}. We avoided the frequent
		use of asynchronous method calls with \emph{Celluloid's} \texttt{async}
		through Fibers (for example for the dispatch method) as it is creating
		an overhead with every request. The Rack framework call to get a
		response to a request also happens inside the main event loop. With
		this way of message handling it is important that the framework or the
		application programmer does not do blocking I/O operations, because
		they stop other requests being processed. In the Rack application or
		framework, non-blocking database adapters have to be used. \ac{Rails}
		for example ships \emph{ActiveJob} as a component to easily integrate
		asynchronous operations since version 4.2.

		\begin{figure}
			\begin{center}
				% {\scalefont{0.75}
					\begin{sequencediagram}
						\newinst   {c}{Client}
						\newinst[2]{s}{\emph{Server} actor}
						\newinst   {o}{\emph{Observe} actor}
						\newinst[1]{a}{Rack application}

						\begin{sdblock}{\emph{Observe tick}}{}
							\begin{call}{o}{respond}{s}{}
								\begin{call}{s}{call}{a}{200, "Bar"}
								\end{call}
							\end{call}

							\mess{o}{2.05, "Bar", observe: 2}{c}
						\end{sdblock}
					\end{sequencediagram}
				% }
			\end{center}
			\caption[Exemplary actor interaction on \emph{Observe
				tick}]{Exemplary actor interaction on \emph{Observe tick}. The
				\emph{Observe} actor receives the Rack application response
				from the \emph{Server} actor and subsequently notifies the
				client about a change of the resource representation.}
			\label{img:server:actor:interaction}
		\end{figure}

		Another use case for a \emph{Celluloid} actor is the management of
		observe \cite{observe} register, deregister and notification
		operations. In \autoref{img:server:actor:interaction}, an exemplary
		communication flow between the \emph{Server} and \emph{Observe} actors
		is shown. The graph describes actions performed to handle observe
		notifications in a certain interval. For a more detailed view on the
		design regarding observe, refer to
		\textref{cha:design:protocol:observe}. The \emph{Observe} actor calls
		the Rack application to determine if the resource representation
		changed. This happens from the \emph{Observe} actor's thread
		concurrently to the \emph{Server} thread. The framework and the
		application programmer have to ensure that every data access is thread
		safe. \ac{Rails} itself is capable of multiple threads concurrently
		calling for a framework response. However, version 4.2 uses a Rack
		middleware called \texttt{Rack::Lock} that locks out concurrent
		framework response calls. This can be deactivated with the Rack
		environment option \texttt{rack.multithread} set to \texttt{true}. We
		will deactivate \texttt{Rack::Lock} and allow concurrent framework
		response calls by default in the server software.
		
		A \emph{GarbageCollector} actor removes obsolete entries from the
		message deduplication cache and the list of observers (see
		\autoref{cha:design:protocol:observe}). The cache is placed as a Hash
		instance variable inside the \emph{Server} actor and can be accessed
		from other actors through messages. This ensures thread safe retrieval
		and update of cache entries. Performance tests during the development
		of the concurrency architecture of the server software showed that
		accessing the cache through \emph{Celluloid} actor messaging has to be
		used thrifty. Placing the cache in a separate actor and performing the
		main lookups and cache inserts via messaging from the \emph{Server}
		actor had a negative performance impact of over 500 requests per second
		(presumably through the messaging overhead). In JRuby 1.7 this overhead
		is even bigger since it implements fibers through \ac{OS}
		threads\footnote{\urlJrubyFiber}. In \ac{MRI} (at least until 2.3) the
		standard library Hash class neither is thread safe nor does it support
		concurrent retrieval or update \cite{ruby-thread2}. With a \ac{GIL}
		assuring only one thread running at a time at least thread safety is
		not a problem \cite{ruby-thread1}. JRuby supports several threads
		running concurrently and implements updating instance variables in a
		thread safe manner\footnote{\urlJrubyThreadsafe}. So in both \ac{MRI}
		and JRuby, the standard library Hash class should suffice when it comes
		to conflict free concurrent access. As described in
		\autoref{cha:background:coap:concurrency}, \emph{Californium} uses a
		Hash class supporting concurrent retrieval and updates for performance
		reasons. Adopting a concurrent Hash class for Ruby may also optimize
		performance. The \emph{thread\_safe} gem\footnote{\urlGemThreadsafe}
		provides different thread safe data structures that are Ruby \ac{VM}
		independent. It includes a hash-like object called
		\texttt{ThreadSafe::Cache} which according to the README file has
		\enquote{much better performance characteristics esp. under high
		concurrency}. However, we could not measure a performance gain compared
		to the Ruby standard library Hash (see
		\href{https://github.com/nning/david/blob/master/experiments/thread\_safe.rb}{\texttt{experiments/thread\_safe.rb}
		in the server's respository}\footnote{\urlDavid}). As structures, both
		a Hash with two element Arrays (endpoint and message ID) as keys, and a
		nested Hash (\texttt{\{endpoint => \{mid => value\}\}}) were
		considered. Although the latter performs better in a detached
		experiment (see
		\href{https://github.com/nning/david/blob/master/experiments/hash\_key.rb}{\texttt{experiments/hash\_key.rb}}),
		we could not measure performance benefits in the overall benchmarking
		of the server (see the \texttt{mid\_cache\_nested\_hash} branch).
		According to the analysis of \ac{MRI} 2.2 in
		\citetitle{ruby-microscope} the performance of arbitrary objects as
		Hash keys depends on the quality of their \texttt{\#hash} instance
		method (see Experiment 7-3 of \cite{ruby-microscope}). Like other
		objects, Arrays inherit a well-working \texttt{\#hash} method from the
		\emph{Object} class.

		% TODO	More general performance tuning tricks

		To keep the overall object count and therefore also the Ruby garbage
		collection overhead low, some objects such as \texttt{Strings}
		containing \ac{HTTP} header names for example are instantiated as
		frozen \texttt{String} constants at load time.

		If the Rack environment key \texttt{rack.multithread} is set to true --
		as we intend to do by default -- the middleware \texttt{Rack::Lock} no
		longer ensures only one thread invoking \texttt{call} on the middleware
		stack. Rack middleware classes writing to instance variables are not
		necessarily thread-safe, because they are only instanciated once per
		process. This is easily fixed by using the \emph{dup.\_call}
		workaround. The original \texttt{call} method is renamed to
		\texttt{\_call} and a new \texttt{call} method invokes \texttt{\_call}
		only on previously duplicated instances of the middleware class. We
		will use this method on any implemented Rack middleware.

	\subsection{\acs{DTLS}}
	\label{cha:design:server:dtls}

		The component abstracting transmissions outlined in
		\textref{cha:design:server:coap} has to be designed in such a way that
		the socket class (\texttt{UDPSocket} or with \emph{Celluloid::IO}
		\texttt{Celluloid::IO::UDPSocket} for unencrypted communication)
		instantiated for communications can be configured. A \ac{DTLS}
		implementation would have to provide an \ac{API} like the Ruby
		\texttt{UDPSocket} class. By default, \texttt{Celluloid::IO::UDPSocket}
		is used, because it is event-driven. To preserve this functionality, a
		drop-in socket class replacement with \ac{DTLS} support would have to
		be modeled after \texttt{Celluloid::IO::UDPSocket}. This way, a server
		with a socket abstracting \ac{DTLS} encrypted packets can be started
		instead of or concurrent to the server used for unencrypted messages.

		Unlike with \ac{HTTP}, with \ac{CoAP} it is not unlikely an identity
		derived from a device's public key or a certificate is used to
		authenticate a peer (see section 9 of \cite{coap}). If the Rack
		application is suitably implemented, it can adopt this identity
		assuming that it is passed in through the Rack environment. A Rack
		environment key such as \texttt{coap.dtls.id} could contain the
		identity in \emph{RawPublicKey} mode or the certificate in
		\emph{Certificate} mode. Also the Rack environment can provide a key
		(for example \texttt{coap.dtls}) which can be used for the app to
		determine the \ac{DTLS} encryption mode.

		Transparent authentication based on \ac{DTLS} would require the
		authentication relationship to be terminated at the server and
		translated into a Rails authentication framework specific (e.g.\ 
		warden\footnote{https://github.com/hassox/warden})
		\texttt{Rack::Session} component. It would also require a mapping
		between \ac{DTLS} identities or certificates and application specific
		subjects which has to be provided by the application itself.

\section{Protocol Translations}
\label{cha:design:protocol}

	To translate between CoAP and the \ac{HTTP} centric Rack interface, the
	specifications for \emph{Cross-Protocol Proxying between \ac{CoAP} and
	\ac{HTTP}} (see section 10 of \cite{coap}) and the \emph{Guidelines for
	\ac{HTTP}-\ac{CoAP} Mapping Implementations} \cite{coap-mapping} can be
	used. However, a \ac{CoAP}-\ac{HTTP} proxy in front of an ordinary
	\ac{HTTP} \ac{Rails} application would not suffice the objectives of this
	work as the framework shall support a \ac{CoAP} specific development and
	therefore should be explicit about \ac{CoAP}. The Cross-Protocol Proxying
	section states that \enquote{since \ac{HTTP} and \ac{CoAP} share the basic
	set of request methods, performing a \ac{CoAP} request on an \ac{HTTP}
	resource is not so different from performing it on a \ac{CoAP} resource}
	(section 10.1 of \cite{coap}) and defines a few departures of behavior of
	the methods GET, POST, PUT and DELETE that have to be implemented
	accordingly. Section 7 of \cite{coap-mapping} defines a mapping of
	\ac{CoAP} to \ac{HTTP} response codes (also see section 6 of
	\cite{http-1}). That, however, can not be applied in this case, because
	the framework's \ac{HTTP} response code has to be mapped to a \ac{CoAP}
	response code. Tables \ref{table:mapping:codes:2xx} and
	\ref{table:mapping:codes:4xx} define mappings in the opposite direction
	(adopting entries from the \ac{CoAP} to \ac{HTTP} mapping table where
	possible). As a fallback, \ac{HTTP} return codes not present in the mapping
	table are translated via division by 100 (e.g.\ \ac{HTTP} \emph{404 Not
	Found} maps to \ac{CoAP} \emph{4.04}). To enable explicit returns of
	\ac{CoAP} response codes from Rack applications, \texttt{Float} values can
	be used. In this case, the mapping has to be skipped.

	\begin{table}[h!]
		\begin{center}
			\begin{tabular}{l|l}
				\ac{HTTP}							& \ac{CoAP} \\
				\hline
				200 OK								& 2.05 Content \\
				202 Accepted						& 2.01 Created \\
				203 Non-Authoritative Information	& 2.05 Content \\
				206 Partial Content					& \scriptsize{Not applicable} \\
				207	Multi-Status					& \scriptsize{Not applicable} \\
				208 Already Reported				& \scriptsize{Not applicable} \\
				226 IM Used							& \scriptsize{Not applicable} \\
			\end{tabular}
		\end{center}
		\caption{Mapping of \ac{HTTP} to \ac{CoAP} return codes (\emph{2xx Success})}
		\label{table:mapping:codes:2xx}
	\end{table}
	
	\emph{Informational} \ac{HTTP} response codes (as defined in section 6.2 of
	\cite{http-1}) do not result in an immediate answer by the server and are
	therefore not translated. As \ac{CoAP} supports no \emph{Redirect}
	\ac{HTTP} response codes, another way has to be found to pass redirection
	information to the client. It would be intransparent for the server to
	follow the redirects and return the redirection target resource
	representation to the client as if there was no redirection. The client
	would have no way to determine if a resource is actually existent. To pass
	the redirection information to the client, the server can translate the
	Rack redirect answer into a \ac{JSON} document. An Example is given in
	\autoref{lst:protocol:redirect}. If conditional requests (see section
	5.10.8 of \cite{coap} and 3 of \cite{http-2}) are implemented, \ac{HTTP}
	\emph{304 Not Modified} would be mapped to \ac{CoAP} \emph{2.03 Valid}. All
	\emph{Server Error} response codes that are not translated directly via
	division (505, 506, and 511) map to the generic \ac{CoAP} \emph{5.00
	Internal Server Error}.

	\begin{figure}[h!]
		\begin{lstlisting}[gobble=6,caption={Exemplary \ac{JSON} redirect description},label=lst:protocol:redirect]
			{
				"code": 302,
				"location": "/example"
			}
		\end{lstlisting}
	\end{figure}
	
	To return \ac{CoAP} codes as accurate as possible, sometimes not only the
	\ac{HTTP} return code but also the request method has to be considered. An
	example for this case is the answer to a \ac{CoAP} DELETE request. Section
	4.3.5 of \cite{http-1} states \enquote{the origin server SHOULD send a 202
	(Accepted) status code if the action will likely succeed but has not yet
	been enacted, a 204 (No Content) status code if the action has been enacted
	and no further information is to be supplied, or a 200 (OK) status code if
	the action has been enacted and the response message includes a
	representation describing the status}. The \ac{CoAP} specification (see
	5.8.4 of \cite{coap}) only demands a \ac{CoAP} \emph{2.02 Deleted} status
	code independent of the success cases. So for a more correct
	implementation, response codes have to be mapped from a tuple of \ac{HTTP}
	method and response code to a \ac{CoAP} response code.

	\begin{table}
		\begin{center}
			\begin{tabular}{l|l}
				\ac{HTTP}							& \ac{CoAP} \\
				\hline
				402 Payment Required				& \scriptsize{Not applicable} \\
				407 Proxy Authentication Required	& 4.01 Unauthorized \scriptsize{(Not really applicable)}\\
				409 Conflict						& 4.12 Precondition Failed \\
				410 Gone							& 4.04 Not Found \\
				411 Length Required					& 4.02 Bad Option \\
				414 URI Too Long					& 4.02 Bad Option \\
				416 Range Not Satisfiable			& 4.02 Bad Option \scriptsize{(Not really applicable)}\\
				417 Expectation Failed				& \scriptsize{Not applicable} \\
				426 Upgrade Required				& \scriptsize{Not applicable} \\
%				422 Unprocessable Entity			& \\
%				423 Locked							& \\
%				424 Failed Dependency				& \\
%				428 Precondition Required			& \\
%				429 Too Many Requests				& \\
%				431 Request Header Fields Too Large	& \\
%				451 Unavailable For Legal Reasons	& \\
			\end{tabular}
		\end{center}
		\caption{Mapping of \ac{HTTP} to \ac{CoAP} return codes (\emph{4xx Client Error})}
		\label{table:mapping:codes:4xx}
	\end{table}

	% Rails provides symbols for \ac{HTTP} return codes to make setting the
	% return code for a response in the controller more
	% readable\footnote{\urlRailsStatusCodes} (for example
	% \texttt{:not\_found}).
	
	Where a direct translation is adequate, the basic mapping of headers will
	be implemented as shown in \autoref{table:mapping:headers}. (This table
	does not consider all existing HTTP headers, because of their huge number.)

	\begin{table}
		\begin{center}
			\begin{tabular}{l|l}
				\ac{CoAP} option								& \ac{HTTP} header \\
				\hline
				Accept											& Accept, Accept-Charset \\
				Content-Format									& Content-Type \\
				ETag \scriptsize{(Not necessarily reversible)}	& ETag \\
				Proxy-Uri, Proxy-Scheme							& \scriptsize{Not applicable} \\
				Max-Age											& Cache-Control: max-age \\
				Location-Path, Location-Query					& Location \\
				If-Match										& If-Match \\
				If-None-Match									& If-None-Match \\
				Size1											& \scriptsize{Not existing} \\
				Size2 \scriptsize{(if block-wise or 4.13)}		& Content-Length \\
			\end{tabular}
		\end{center}
		\caption{Direct mappings from \ac{CoAP} options to \ac{HTTP} headers}
		\label{table:mapping:headers}
	\end{table}

	% TODO	Explain ETag translation

	Some translations could be solved through Rack middleware (see
	\autoref{cha:design:server:rack}). Except for the Resource Discovery,
	translation through methods present directly in the server code is favored
	over middleware usage. Using middleware for translating \ac{HTTP} to
	\ac{CoAP} headers would change the Rack environment to contain \ac{CoAP}
	headers. This would break transparency of that exemplary middleware and it
	will not be possible to prepend it by another middleware expecting
	\ac{HTTP} headers.

	The following subsections describe the translation of more complex protocol
	features.

	\subsection{Block-wise Transfers}

		The most naive implementation of handling requests for certain blocks
		of a resource representation through block-wise transfers \cite{block}
		would be requesting a full framework answer on each request and
		returning only the requested chunk. However, if the resource
		representation changes in between different requests with Block2 Option
		in control usage (see section 2.3 of \cite{block}), the payload
		possibly can not be reassembled on the client side.  According to
		section 2.4 of \cite{block} a server should include an ETag option with
		its responses. This way a client can try to obtain a fully consistent
		resource representation again if the ETag value suddenly changes. It is
		also possible to cache the resource representation initially received
		from the framework in connection with the endpoint until the last chunk
		is requested. If the server implements caching of framework responses
		(and the application supports it by setting caching headers), the
		performance can be increased, because the resource representation could
		be cached for all endpoints.  Another approach would be using \ac{HTTP}
		Range requests \cite{http-3} but that also would have to be supported
		manually by the application\footnote{As of Rails 4.2: \texttt{curl -sI
		http://localhost:3000 | grep Accept-Ranges}}. For now we intend to use
		the naive implementation, because it does not require developing effort
		on the Rack application side.

	\subsection{Content Negotiation and Transcoding}
	\label{cha:design:protocol:content}

		For the Content Negotiation to be translated between both protocols
		(see section 5.3 of \cite{http-1} for \ac{HTTP} and section 5.5.4 of
		\cite{coap} for \ac{CoAP}), the accept option of a \ac{CoAP} request
		has to be rewritten into a \ac{HTTP} media type and a content encoding,
		which can be passed in to the Rack application through the
		\texttt{HTTP\_ACCEPT} field in the environment. If possible, the
		application responds with an appropriate body representation and the
		correctly set \texttt{Content-Type} header. Section 6.2 of
		\cite{coap-mapping} further specifies the media type mapping and
		defines a \emph{Loose Media Type Mapping} that can be used to
		generalize \ac{HTTP} media types to ones supported by \ac{CoAP}. In
		section 6.3 also an algorithmic conversion from \ac{HTTP} Internet
		Media Type to \ac{CoAP} Content Format is given.

		For constrained nodes, a compact payload serialization format such as
		\ac{CBOR} \cite{cbor} is essential. It is possible to transcode between
		payload content formats to make use of efficient formats (as mentioned
		in section 6.4 of \cite{coap-mapping}). \ac{JSON} can be mapped
		transparently to \ac{CBOR}. This is particularly useful, because
		\ac{JSON} is easily generated out of a \ac{Rails} 4 application by
		direct conversion of an object (with the \texttt{to\_json} instance
		method) from the controller or a \texttt{jbuilder} template as view
		specification. To support formats that can not directly be converted to
		\ac{JSON}, conversion methods and possibly a \ac{DSL} for easier
		templating would be needed, therefore this work currently focuses
		solely on \ac{CBOR}.

		The cbor gem \cite{cbor-gem} is a ready-to-use library \enquote{based
		on the (polished) high-performance msgpack-ruby code} for serialization
		of Ruby objects into \ac{CBOR} and vice versa. Although the gem
		currently does not support JRuby as a Ruby \ac{VM}, we favor it over
		cbor-simple\footnote{\urlCborSimple}, because of its performance.

		Transcoding of incoming requests with \ac{CBOR} payloads to \ac{JSON}
		is less straightforward. Other than \ac{JSON}, \ac{CBOR} does not only
		support Strings as keys in Hashes. When converting a Hash to \ac{JSON},
		the standard library \texttt{JSON.parse} method calls \texttt{\#to\_s}
		on the keys. An exemplary Hash \texttt{\{1 => 2\}} decoded from
		\ac{CBOR} would become the \ac{JSON} String \texttt{'\{"1" => 2\}'}. So
		if we actually want to pass the information that the key \texttt{1} is
		a Fixnum correctly into the controller, we can not use \ac{JSON}. A
		method would have to be found how to convert incoming \ac{CBOR} payload
		directly into the Rails \texttt{params} Hash as it is done with
		incoming \ac{JSON} data by Rails. A less elegant way would be letting
		the application programmer parse the \ac{CBOR} body in the controller.
		Handing down a Ruby object deserialized from \ac{CBOR} in the Rack
		environment (for example as value of \texttt{coap.cbor}) would provide
		a little convenience and prevent redundant parsing.

		When automatically translating \ac{JSON} bodies to \ac{CBOR}, protocol
		compatibility and transparency have to be preserved. If the client does
		not specify a content format preference through the accept option, a
		\ac{JSON} response of the Rack application can be translated
		transparently into \ac{CBOR}. The same applies if the client specifies
		\ac{CBOR} as its preferred content format. If the client actually
		requests the body represented in a certain content format different
		from \ac{CBOR} (e.g.\ \ac{JSON}), the answer must also be in that
		content format to not break with Content Negotiation as required by
		\ac{CoAP} or \ac{HTTP}.

		Transcoding could be implemented as a Rack middleware that is also a
		\emph{Celluloid} actor. A message containing information about which
		Media Types for which controller action are to be transcoded could be
		sent from the Rails controller to the \texttt{Transcoding} middleware
		actor. This would enable the application developer to decide for which
		controller and actions a configurable transcoding happens. However,
		currently we know of no way to determine the controller and action in a
		middleware \texttt{call}. Without this information, a transcoding
		middleware has no advantages over transcoding the body representation
		outside of the Rack middleware stack.

	\subsection{Chunked Transfer Coding}

		Since version 1.5.0, Rack supports \ac{HTTP}/1.1 Chunked Transfer
		Coding (see section 4.1 of \cite{http-0}). In a plain Rack application
		as seen in \textref{lst:rack:basic} for example, the response body is
		chunked transparently if the \emph{Content-Length} header is omitted.
		There is not really a \ac{CoAP} extension which is comparable, although
		with \cite{coap-streaming} there is an expired draft. This, however,
		needs the client to explicitly send an observe register (see sections
		2.1 of \cite{coap-streaming} and 3.1 of \cite{observe}). The server has
		to reassemble the chunked resource representation, possibly cache it
		and return it to the client.

	\subsection{Resource Discovery}

		The \ac{CoAP} Resource Discovery (section 7 of \cite{coap}) is
		especially useful in \ac{M2M} communication. After nodes are discovered
		through Service Discovery (see section 7.1), the
		\texttt{.well-known/core} interface according to \cite{link} (as
		referenced by section 7.2) can be used to discover a node's resources
		and their attributes such as the content format or an interface
		description. An automatic listing of a rack applications resources is
		examined in the next subsection. The Multicast subsection describes the
		requirements of multicast enabled resource discovery. The design of a
		\acf{RD} \cite{rd}, which maintains \ac{CoRE} Link collections of
		different nodes, is considered in the \acl{RD} subsection.

		\subsubsection{Reflection of Routing in .well-known/core}
		\label{cha:design:protocol:link}

			The \texttt{.well-known/core} resource returning a description of
			the available resources of the Rack application in the \ac{CoRE}
			Link Format \cite{link} can be provided by the \ac{CoAP} server as
			a Rack middleware. By including the server into a \ac{Rails}
			project, a middleware can be inserted into the stack automatically
			utilizing a Railtie \cite{railtie}. There is no general way to list
			the resources provided by a Rack application or an application
			written in any Rack based web framework. Using \ac{Rails}, an
			enumeration of the defined routes is possible by processing
			\texttt{Rails.application.routes}. To support different Rack based
			web frameworks, it is necessary to create framework specific
			modules for the enumeration of resources.

			The only attribute of a Link that can be determined by just
			enumerating the routes is \texttt{href}. A method of annotating
			resources with information about Resource Type (rt), Interface
			Description (if), Maximum Size Estimate (sz), and the other
			possible attributes for CoRE Link descriptions has to be developed.
			One possibility would be providing a method in the context of a
			controller defining both default attributes and attributes for
			specific resources. \autoref{lst:discovery:annotation} provides an
			example of how this could be used. On call of the
			\texttt{discoverable} method, these attributes have to be
			communicated to the instance of the Resource Discovery Rack
			middleware, which is instantiated at the start of the server.
			Besides dynamically changing a method of the middleware instance to
			return the right structure containing any defined link attributes,
			the middleware instance could also be a \emph{Celluloid} actor that
			can receive messages triggered by the call of \texttt{discoverable}
			in the controller context. An overview in the form of a sequence
			diagram is to be seen in \autoref{img:discovery:annotation}.
			Possible performance impacts of \emph{Celluloid} actor inclusion in
			the middleware stack have to be analyzed.

			\ac{Rails} returns the available resources as \ac{URI} patterns.
			Those patterns are known for example from the output of the command
			\texttt{rake routes} ran in a \ac{Rails} project. An \ac{URI}
			pattern like \texttt{/things/:id} could be machine processable, if
			all involved endpoints support \texttt{:id} is a variable
			substitution. But this pattern follows no known standard. The
			\ac{URI} Template standard \cite{uri-template} defines ways of
			variable expansions for \acp{URI}. By this definition, the
			previously mentioned pattern could be represented by the template
			\texttt{/things/\{id\}}. This substitution can easily be performed
			by the middleware. However, \ac{URI} Templates are not supported in
			\ac{CoRE} Link \ac{URI}-references.
 
			Actors included in the middleware stack should be supervised and
			therefore restarted on crash, because otherwise any request fails
			after one actor in the stack died due to the impossibility of
			invoking the \texttt{call} method on a dead actor. This is not
			trivial, because the \emph{Celluloid} \ac{API} only provides a
			supervision possibility if the actor is instantiated in a special
			way but Rack expects to instanciate middleware objects itself.
			Probably this can be solved through a proxy object.

			% TODO	Analyze actor middleware performance somewhere.
			
			\begin{figure}
				\begin{lstlisting}[gobble=8,caption={Action annotation for Resource Discovery},label={lst:discovery:annotation}]
					class ThingsController < ApplicationController
						discoverable \
							default: { if: 'urn:things', ct: 'application/cbor' },
							index:   { if: 'urn:index' }

						def show
							render json: Thing.find(params[:id])
						end

						def index
							render json: Thing.all
						end
					end
				\end{lstlisting}
			\end{figure}

			\begin{figure}
				\begin{center}
					% {\scalefont{0.75}
						\begin{sequencediagram}
							\newinst   {c}{Client}
							\newinst[2]{s}{\emph{Server} actor}
							\newinst   {d}{\emph{Discovery} actor}
							\newinst[1]{a}{Controller contexts}

							\mess{a}{Link attributes}{d}

							\begin{messcall}{c}{GET /.well-known/core}{s}
								\begin{call}{s}{call}{d}{}
								\end{call}

								\mess{s}{2.05}{c}
							\end{messcall}
						\end{sequencediagram}
					% }
				\end{center}
				\caption[Interaction regarding action annotation for Resource
					Discovery]{Interaction regarding action annotation for
					Resource Discovery. The \emph{Discovery} actor receives
					link attributes of resources from the respective
					controller. The resource descriptions returned on client
					request then contain these attributes.}
				\label{img:discovery:annotation}
			\end{figure}

		\subsubsection{Multicast}
		\label{cha:design:protocol:multicast}

			The \texttt{.well-known/core} resource has to be offered
			additionally on multicast requests to the addresses defined in
			sections 8 and 12.8 of \cite{coap} and section 2.2 of
			\cite{coap-group}. The implementation guidance document draft
			\cite{coap-guide} does not state anything regarding the
			implementation of multicast Resource Discovery. To limit the size
			and number of answers, it is beneficial to support filtering the
			result set by the entries attributes. If the result set is empty,
			no response should be sent on multicast requests (as it is
			described by section 8.2 of \cite{coap} and section 4.1 of
			\cite{link}. Through a \ac{CoAP} specific Rack environment entry
			(\texttt{coap.multicast}), it would be possible for applications to
			determine if an incoming request was received on a multicast
			address.

			% TODO	It has to be evaluated..!

			It has to be evaluated how the synchronous nature of \ac{HTTP} and
			therefore Rack and \ac{Rails} can be integrated for example with
			the leisure period requirement mentioned in section 8.2 of
			\cite{coap}.

		\subsubsection{\acf{RD}}
		\label{cha:design:protocol:rd}

			A \acl{RD} \cite{rd} \enquote{hosts descriptions of resources held
			on other servers, allowing lookups to be performed for those
			resources}. The specification can be implemented as a \ac{Rails}
			application or mountable engine \cite{rails-engine}. The provision
			of the \texttt{.well-known/core} resource from a Rack middleware
			(as specified earlier in this section) will not be used in this
			case, because it is integrated with the routing and controllers of
			the \ac{Rails} application. As the default content format is
			\texttt{application/link-format}, we need a way to make it
			configurable for the application.

	\subsection{Observe}
	\label{cha:design:protocol:observe}

		The \ac{CGI} inspired design of Rack aimed at synchronous
		communications complicates the integration of asynchronous concepts
		like \emph{WebSocket} or -- in our case -- \ac{CoAP} observe
		\cite{observe}. \emph{Rack Hijack} \cite{rack} describes the proceeding
		of handing over the socket object from the server to the application
		(controllers) as a Rack environment option. For \emph{WebSockets},
		there is a gem called \emph{tubesock}\footnote{\urlTubesock} utilizing
		\emph{Rack Hijack}, which could be used as an inspiration for a gem
		that provides support for \emph{Observe} in \ac{Rails} in a similar
		manner. However, this is rather bad design, because placing the observe
		logic out of the server and into an extra gem used from the framework
		facilitates code duplication and breaks the model of a layered stack of
		abstractions from a object-oriented design point of view. In
		\emph{tubesock}, this may even be a valid design choice, because the
		\emph{WebSocket} protocol \cite{websocket} is out-of-band and not
		handled by the main web server. The code to be executed is handed over
		to \emph{tubesock} by callbacks, which conflicts with the declarative,
		object-oriented design philosophy of \emph{Celluloid}. This method also
		circumvents the Rack middleware stack.

		Another approach would be viewing a component handling observe
		notifications and observer management as an actor (realized with
		\emph{Celluloid}). On incoming observe register or deregister requests,
		the server actor(s) message the observe actor. The observe actor
		maintains the list of registered observers and periodically determines
		whether to send observe notifications (further called \emph{Observe
		tick}). The interaction of actors among each other and with the client
		and the Rack application in the cases register, deregister and tick are
		visualized in \autoref{img:server:actor:interaction}. The most quick
		and easy implementation for determining whether an update is necessary,
		is to clone the original request of a resource, pass it to the
		framework (Rack application) and send a notification if the resource
		changed. With the framework answer's validator header fields
		\texttt{Last-Modified} and \texttt{ETag} (see sections 2.2 and 2.3 of
		\cite{http-2}) (or, if they are missing, a hash sum of the body) it can
		be determined if the framework returned a changed body and thus a
		client has to be notified about the updated resource. An optimization
		for frameworks supporting validator header fields is to facilitate
		conditional requests. This way it is possible, the overhead of
		regenerating the resource representation can be stinted. For an
		exemplary interaction in this case, refer to
		\autoref{img:observe:cache}. \ac{Rails} uses \emph{Rack::ETag} for
		entity tags which is a middleware that can also be used to provide
		plain Rack applications or other Rack based frameworks with
		\texttt{ETag} support. In \ac{Rails}, the decision whether the resource
		representation has to be rendered is made from controller code and
		thereby explicitly necessary. The notification can be omitted if the
		entity tag did not change and the max-age time of the last notification
		is not expired. If this max-age time is expired and the entity tag did
		not change, a notification with \ac{CoAP} \emph{2.03 Valid} response
		code is sent.
		
		For the observer list management, besides adding observers to resources
		and deleting them, the observe actor has to implement a
		\enquote{garbage collection} for obsolete observe relationships (as
		described in section 3.6 of \cite{observe}). If observe notifications
		are always sent as non-confirmable messages and the server does not
		send any other confirmable messages, the retransmission logic can be
		omitted. This saves some memory and \ac{CPU} resources. The
		specification explicitly states, \enquote{A notification can be
		confirmable or non-confirmable, i.e., it can be sent in a confirmable
		or a non-confirmable message. The message type used for a notification
		is independent of the type used for the request and of any previous
		notification.} (see section 3.5 of \cite{observe}).

		\begin{figure}
			\begin{center}
				% {\scalefont{0.75}
					\begin{sequencediagram}
						\newinst   {c}{Client}
						\newinst[2]{s}{\emph{Server} actor}
						\newinst   {o}{\emph{Observe} actor}
						\newinst[1]{a}{Rack application}

						\begin{sdblock}{\emph{Observe register}}{}
							\begin{messcall}{c}{GET /example, observe: 0}{s}
								\begin{call}{s}{call}{a}{200, "Foo", ETag: 0x91}
									\postlevel
								\end{call}

								\mess{s}{add}{o}

								\mess{s}{2.05, "Foo", observe: 1, etag: 0x91}{c}
							\end{messcall}
						\end{sdblock}

						\begin{sdblock}{\emph{Observe tick}}{}
							\begin{messcall}{o}{respond}{s}
								\begin{call}{s}{call(If-None-Match: 0x91)}{a}{304 Not Modified}
								\end{call}
							\end{messcall}

							\begin{messcall}{o}{respond}{s}
								\begin{call}{s}{call(If-None-Match: 0x91)}{a}{200, "Bar", ETag: 0x63}
									\postlevel
								\end{call}
							\end{messcall}

							\mess{o}{2.05, "Bar", observe: 2, etag: 0x63}{c}
						\end{sdblock}
					\end{sequencediagram}
				% }
			\end{center}
			\caption[Observe interaction with conditional requests]{Observe
				interaction with conditional requests. On \emph{Observe
				register} the \texttt{ETag} value of the Rack response is saved
				and later used to determine if the resource representation
				changed and an update is necessary.}
			\label{img:observe:cache}
		\end{figure}

		For the application code it would be possible to utilize the HTTP/1.1
		caching headers like \texttt{Expires} or \texttt{Cache-Control max-age}
		(see sections 5.3 and 5.2 of \cite{http-4}) to communicate when a
		resource should be polled again and in which interval a resource change
		should be polled for. Polling for resource updates wastes resources, so
		for a more performant solution, the framework or application code would
		have to message the observe actor about a resource update. If the
		resource is a representation of a list of saved entities for example,
		the framework would have to notify the observe actor on each creation
		or deletion of an entity. However, this problem can not generally be
		solved by callbacks triggered on database changes. Explicitly
		signalling a resource update with a certain method call from the
		\ac{Rails} controller is not possible, since the controller code is
		only called upon request from the server or observe actors and not
		running as an actor itself.

		In \autoref{lst:observe:example} a \ac{Rails} controller code
		example is shown that both manipulates the polling frequency of the
		observe actor for this resource and only renders the resource
		representation if a conditional request by the observe actor indicates
		a change. \emph{\ac{Rails}::Observers}\footnote{\urlRailsObservers} can
		be used to further control cache invalidation.

		% TODO	Examine Rails Observers
		% 		https://github.com/rails/rails-observers

		\begin{figure}
			\begin{lstlisting}[gobble=8,caption={Caching options in \ac{Rails} controller},label={lst:observe:example}]
				class ThingsController < ApplicationController
					def show
						@thing = Thing.find(params[:id])

						# Set Cache-Control max-age to 60 seconds.
						# Manipulates observe actor polling frequency.
						expires_in 1.minutes

						# Set Last-Modified and ETag headers
						# and respond only if conditional request matches.
						render json: @thing if stale? @thing
					end
				end
			\end{lstlisting}
		\end{figure}

% \section{Conclusion}

% We tried to exhaustively design most aspects of the software system but
% obviously there are limits to the level of detail possible in the context of
% this work. Further features can iteratively be added to the software, because
% of its modularity.
