\chapter{Implementation}
\label{cha:implementation}

This chapter documents the implementation drafted in the previous chapter.
Each component realized as an independent module is explained in detail in the
following sections. \autoref{table:implementation:components} gives an overview
of the components we worked on, their respective licenses, and the sections
covering their implementation.

\begin{table}[H]
	\centering
	\begin{tabular}{l|l|l}
		Component & License & Section\\
		\hline
		coap (Library) & MIT & \ref{cha:implementation:client} \\
		david (Server) & GPL & \ref{cha:implementation:david} \\
		core-rd (\ac{RD}) & AGPL & \ref{cha:implementation:core-rd} \\
	\end{tabular}
	\caption{Implemented Components}
	\label{table:implementation:components}
\end{table}

\section{External Libraries}

	For all implemented components, we make use of several external libraries.
	The most important ones are listed in
	\autoref{table:implementation:libraries}. A more exhaustive list can be
	obtained from the \texttt{Gemfile.lock} files inside the project
	repositories.

	\begin{table}[H]
		\centering
		\begin{tabular}{l|l|l}
			Library			& Version	& License \\
			\hline
			cbor			& 0.5.9		& Apache 2.0 \\
			celluloid		& 0.16		& MIT \\
			celluloid-io	& 0.16.2	& MIT \\
			rack			& 1.6		& MIT \\
			rails			& 4.2		& MIT \\
			rspec			& 3.2		& MIT \\
		\end{tabular}
		\caption{External Libraries}
		\label{table:implementation:libraries}
	\end{table}

	A small change was integrated into the \emph{Celluloid::IO} project.
	Through inclusion of \emph{Celluloid::IO}, socket classes are wrapped in a
	way that prevents blocking usage. Methods which do not block are directly
	delegated to the wrapped standard library socket class. A delegation for
	the \texttt{addr} instance method was added in the course of this
	work\footnote{\urlCelluloidIoPr}.

\section{\acs{CoAP} Library}
\label{cha:implementation:client}

	The \ac{CoAP} library based on the message parser by Carsten Bormann
	published by the GOBI project\footnote{\urlGobiCoap} was forked on
	GitHub\footnote{\url{https://github.com/nning/coap}}. The original project
	is released under the terms of the MIT
	license\footnote{\url{https://github.com/SmallLars/coap/blob/master/LICENSE}},
	and so is the fork we developed. One of the original authors, Simon
	Frerichs, gave us access to the coap gem on rubygems.org and the first
	version of our fork we released was
	0.1.0\footnote{\url{https://rubygems.org/gems/coap}}. The current release
	version is 0.1.1. Initial objectives were: Restructurization,
	modularization (for code reuse in the server component), removal of code
	duplication, freeing of tests from external services, and bug fixes.
	Another important objective was to make any socket communications
	non-blocking and shared data structures compatible with concurrent access.
	The following lists extracted from the commit log summarize the individual
	changes to the \ac{CoAP} library\footnote{\urlCoapChanges} by category. The
	whole commit log in comparison to the original source code can be found on
	GitHub\footnote{\url{https://github.com/nning/coap/compare/SmallLars:master...master}}.

	% TODO	Describe most important changes as text with code snippets
	%		(especially Transmission)

	\begin{multicols}{2}
		\subsubsection{Refactoring}

			\begin{itemize}
				\item Refactoring and simplification of tests
				\item Refactoring of socket abstraction class
				\item Modularization of message parser
				\item Refactoring of central client method (especially method
					extraction)
			\end{itemize}

		\subsubsection{Janitor Tasks}

			\begin{itemize}
				\item Update message parser
				\item \ac{CoRE} namespace
				\item More current Ruby version on Travis CI
				\item Gem updates
			\end{itemize}

		\subsubsection{Code Optimization and Reimplementation}

			\begin{itemize}
				\item Reimplementation of command line utility
				\item Unification and better documentation of client method
					arguments
				\item Bugfixes in message parser (path and query encoding)
				\item \texttt{\ac{CoAP}::Message\#to\_s}
				\item Preparations for JRuby compatibility
				\item Content format registry
				\item \ac{CoRE} Link format \cite{link} implementation
				\item Changed client method argument order from \{host, port,
					path, method, payload, options, callback\} to \{method,
					path, host, port, payload, options, callback\} (all but
					method and path are omitable)
				\item Changed socket abstraction to non-blocking transmission
					layer class (\texttt{Transmission}) based on
					\texttt{Celluloid::IO}
				\item Separate support and ACK on CON in
					\texttt{Transmission} class
				\item JRuby support
				\item Optimized chunkifying payload for block-wise requests
				\item Corrected maximum random numbers for mid and token
				\item Configurable socket for \texttt{Transmission}
				\item \ac{CoAP} transmission \acp{FSM} (not yet properly
					integrated)
				\item Reimplementation of \texttt{Block} class
				\item Utility code to determine \acl{OS}, check if network
					interface is up and network interface name to index
					conversion (needed in context of server multicast
					implementation)
				\item Support for setting the query in client
			\end{itemize}

		\subsubsection{Testing}

			\begin{itemize}
				\item Preparations for RSpec \cite{rspec}
				\item Porting of client tests to RSpec (block1, observe, and
					separate)
			\end{itemize}
	\end{multicols}

\section{David}
\label{cha:implementation:david}

	The core of this work -- the CoAP server component with a Rack interface --
	is called \emph{David} in reference to
	\href{http://postrank-labs.github.io/goliath}{Goliath} (an asynchronous
	\ac{HTTP} server with a Rack interface). It is realized as a Ruby
	gem\footnote{\url{https://rubygems.org/gems/david}} and can be used as a
	drop-in replacement for \ac{HTTP} servers in a \ac{Rails} application. It
	is released under the terms of the \ac{GPL} \cite{gplv3} and downloadable
	on GitHub\footnote{\urlDavid}. James Fairbairn originally held \emph{david}
	as a name on rubygems.org but kindly gave it to us. The first public
	release was version 0.3.0.pre. The current stable release is 0.4.3 and all
	code descriptions in this section refer to that version.

	The code was developed mostly on \ac{MRI} versions between 2.1.4 and 2.2.0
	but it was also optimized for and tested on \ac{MRI} 1.9.3-p551 and
	2.3.0-dev, JRuby 1.7.18 and 9.0.0.0-pre, and Rubinius from 2.4.0 till
	2.5.2. In JRuby and Rubinius, the detection of multicast messages does not
	work because of the platforms' \texttt{Socket} \acp{API} (for further
	details, see \autoref{cha:implementation:david:protocol}). The message
	handling of David running on Rubinius suffers from a deadlock problem that
	could not been solved yet. Some of the server's features such as
	application Resource Discovery (see \autoref{cha:implementation:discovery})
	are specific or at least tied to Rails and do not work with every Rack
	based framework.
	
	The following subsections handle specific aspects of the implementation.

	\subsection{Options}

		Besides configuration through the commandline (e.g.\ with the
		\texttt{-O} option of \texttt{rackup}), the server's options can be set
		from the Rails application config (e.g.\ in
		\texttt{config/application.rb}. The possible configuration options are
		listed in \autoref{table:implementation:david:options}.

		\begin{table}[H]
			\centering
			\begin{tabular}{l|l|l|l}
				Rack key & Rails key & Default & Semantics \\
				\hline
				Block & coap.block & true & Block-wise transfer support \\
				\ac{CBOR} & coap.cbor & false & \ac{JSON}/\ac{CBOR} transcoding \\
				DefaultFormat & coap.default\_format & & Default Content-Type \\
				& & & (if \ac{CoAP} accept omitted) \\
				Host & & ::1 / :: & Server listening host \\
				Log & & info & Log level (none or debug) \\
				MinimalMapping & & false & Minimal mapping \\
				& & & (see \autoref{cha:evaluation:interoperability:rack}) \\
				Multicast & coap.multicast & true & Multicast support \\
				Observe & coap.observe & true & Observe support \\
				& coap.only & true & Removes \ac{HTTP} middleware \\
				& & & and adds David as default \\
				& & & Rack handler \\
				Port & & 5683 & Server listening port \\
				& \small{coap.resource\_discovery} & true & Provision of \\
				& & & \texttt{.well-known/core} \\
			\end{tabular}
			\caption{David configuration options}
			\label{table:implementation:david:options}
		\end{table}

	\subsection{Rack}

		David adds some application specific entries to the Rack environment
		which are listed in \autoref{table:implementation:david:rack}.

		\begin{table}[H]
			\centering
			\begin{tabular}{l|l|l}
				Key				& Value class	& Semantics \\
				\hline
				coap.version	& Integer		& Protocol version of \ac{CoAP} request \\
				coap.multicast	& Boolean		& Marks whether request was received via multicast \\
				coap.dtls		& String		& \ac{DTLS} mode (as defined in section 9 of \cite{coap}) \\
				coap.dtls.id	& String		& \ac{DTLS} identity \\
				coap.cbor		& Object		& Ruby object deserialized from \ac{CBOR}
			\end{tabular}
			\caption{Rack environment additions of David}
			\label{table:implementation:david:rack}
		\end{table}

		We monkey-patched the Rack code to use David as the default Rack
		handler unless Rails is loaded and \texttt{config.coap.only} is set to
		\texttt{false} (see
		\href{https://github.com/nning/david/blob/0.4.3/lib/david/guerilla/rack/handler.rb}{\texttt{lib/david/guerilla/rack/handler.rb}}).
		This way executing \texttt{rackup} or \texttt{rails s} starts the
		\ac{CoAP} server, and does not try the original Rack handler order
		(thin, puma, webrick). The actual Rack handler is defined in
		\href{https://github.com/nning/david/blob/0.4.3/lib/rack/handler/david.rb}{\texttt{lib/david/rack/handler/david.rb}}
		and performs the startup of the different server actors
		(\texttt{Server}, \texttt{GarbageCollector}, and \texttt{Observe}) in a
		\emph{Celluloid} supervision group.

		Automatic provision of Resource Discovery trough a
		\texttt{.well-known/core} resource and return of exceptions caused by
		the application in a human readable format are created as Rack
		middleware classes \texttt{David::ResourceDiscovery} (see
		\autoref{cha:implementation:discovery}) and
		\href{https://github.com/nning/david/blob/0.4.3/lib/david/show\_exceptions.rb}{\texttt{David::ShowExceptions}
		(see \texttt{lib/david/show\_exceptions.rb})}. They are inserted into
		the middleware stack by the \texttt{David::Railties::Middleware}
		railtie (see
		\href{https://github.com/nning/david/blob/0.4.3/lib/david/railties/middleware.rb}{\texttt{lib/david/railties/middleware.rb}}).
		Other middleware shipped with Rails is removed if it (currently)
		provides no features in connection with \ac{CoAP} (see line 8 ff.). It
		is possible to keep it included in the middleware stack by setting
		\texttt{coap.only} to \texttt{false} in the Rails application config.

	\subsection{Protocol Implementation}
	\label{cha:implementation:david:protocol}

		Parsing incoming transmissions does not utilize other shared
		functionality from the \ac{CoAP} library than the
		\texttt{CoAP::Message.parse} method, which was provided by Carsten
		Bormann and only changed minimally. For answering or initiating
		transmissions, the \texttt{CoAP::Transmission} class (see
		\href{https://github.com/nning/coap/blob/0.1.1/lib/core/coap/transmission.rb}{\texttt{lib/core/coap/transmission.rb}}
		in version 0.1.1 of the coap gem source) was used first to avoid code
		duplication. It abstracts timeouts, retransmissions, ACK answering,
		separate answers and checking of message identifiers and tokens. For
		more information, refer to \autoref{cha:implementation:client}.
		However, when optimizing the server for throughput performance it
		became obvious that this shared functionality also introduced overhead.
		Most features were too client centric anyway: Retransmissions, incoming
		separate transmissions and message token management are not necessarily
		useful for a server. These shared code parts are also not modular
		enough to allow a single place for incoming message dispatching. So
		after a redesign of the architecture, messages are sent by direct
		invocation of \texttt{send} on the server socket. References to that
		socket are passed between actors. The first version with the revised
		architecture is 0.4.0. A cache for message correlation and
		deduplication of confirmable requests is currently solved through a
		Ruby hash with endpoint and message identifier as key and the cached
		response and a timestamp for cache invalidation as values (see
		\href{https://github.com/nning/david/blob/0.4.3/lib/david/server/mid\_cache.rb}{\texttt{lib/david/server/mid\_cache.rb}}).
		The \texttt{GarbageCollector} actor (see
		\href{https://github.com/nning/david/blob/0.4.3/lib/david/garbage\_collector.rb}{\texttt{lib/david/garbage\_collector.rb}})
		periodically cleans the cache from obsolete messages. Using the
		reimplemented block support of the coap gem, block-wise transfers
		\cite{block} were integrated into the server. However, the current
		state only supports block-wise responses; block-wise transfers for
		assembling requests are not supported. Separate responses as designed
		in \autoref{cha:design:server:coap:separate} were not implemented yet
		as they make a bigger change to the current request/response cycle
		necessary.
		
		Observe support is accomplished through a \emph{Celluloid} actor
		handling the notifications. The \texttt{Server} actor adds or removes
		listeners via \texttt{\#add} or \texttt{\#delete} messages to the
		\texttt{Observe} actor (as depicted in \autoref{img:observe:cache};
		also see
		\href{https://github.com/nning/david/blob/0.4.3/lib/david/server/respond.rb#L133}{line
		133 of \texttt{lib/david/server/respond.rb}}). Listeners are saved in a
		Ruby hash with endpoint and message token as key and observe number
		(initially 0), original request, rack environment, originally returned
		\texttt{ETag}, and a timestamp for cache invalidation as values (see
		\href{https://github.com/nning/david/blob/0.4.3/lib/david/observe.rb}{\texttt{lib/david/observe.rb}}).
		In a static interval of 3 seconds by default, the \texttt{Observe}
		actor checks every target resource of the observe relationships on
		changes and sends updates to the endpoint if necessary. The update
		check is solved by requesting a full framework answer for the original
		request from the \texttt{Server} actor. If the \texttt{ETag} differs
		between the original request and response, an observe notification is
		sent to the endpoint. The observe relationship persists unless the
		endpoint answers with a reset message to a notification, explicitly
		deregisters or the update check is answered with an error response code
		by the framework or application. There is potential for performance
		improvements in the current implementation, because the
		\texttt{Observe} actor neither refreshes every individual resource only
		once per tick nor makes effective use of \ac{HTTP} caching as drafted
		in \autoref{cha:design:protocol:observe}. A correctly set \texttt{ETag}
		header is mandatory for the \texttt{Observe} actor to decide if an
		observe notification has to be sent. Rails automatically includes
		\texttt{Rack::ETag}, which is a Rack middleware that transparently adds
		an \texttt{ETag} header. This middleware can be included manually when
		using more minimal frameworks. As an example, see
		\href{https://github.com/nning/david/blob/0.4.3/config.ru#L6}{line
		6 of \texttt{config.ru}}, a Rack configuration that starts a plain Rack
		application for testing purposes.

		\autoref{lst:david:eventloop} shows the implementation of the event
		loop. Inside a class including \texttt{Celluloid::IO},
		\texttt{recvfrom} automatically uses the \texttt{Celluloid::IO} event
		loop. There were some changes necessary to the original message
		handling to determine if a request has been received on a multicast
		address. The non-blocking \texttt{\#recvfrom} method exposed by
		\texttt{Celluloid::IO::UDPSocket} class instances does not return
		sufficient information. Therefore \texttt{\#recvmsg\_nonblock} of the
		wrapped \texttt{UDPSocket} class instance was used to receive data and
		information on the target address from the socket. The event loop was
		implemented as in \texttt{\#recvfrom} of
		\texttt{Celluloid::IO::UDPSocket} (see
		\href{https://github.com/celluloid/celluloid-io/blob/9ccce9ba40535b4fa204e69f312bc2fe53dda3a1/lib/celluloid/io/udp_socket.rb#L19}{the
		\texttt{Celluloid::IO::UDPSocket} implementation} and
		\href{https://github.com/nning/david/blob/0.4.3/lib/david/server.rb#L45}{line
		45 of \texttt{lib/david/server.rb}} for reference) using
		\texttt{Celluloid::IO.wait\_readable} which is based on the \ac{OS}
		independent I/O selector \ac{API} of nio4r\footnote{\urlNio}. In JRuby
		and Rubinius we use the non-blocking \texttt{recvfrom} method provided
		by \emph{Celluloid::IO}. This change was not proposed to the
		\emph{Celluloid::IO} project, because we could not get a Ruby \ac{VM}
		independent implementation working due to the absence of a
		\texttt{recvmsg\_nonblock} method in JRuby and Rubinius. Since
		\emph{Celluloid::IO} aims on \ac{VM} independence, this effort was
		canceled for now.

		\begin{figure}
			\begin{lstlisting}[gobble=8,caption={Server Event Loop},label={lst:david:eventloop}]
				loop do
					if jruby_or_rbx?
						dispatch(*@socket.recvfrom(1152))
					else
						begin
							dispatch(*@socket.to_io.recvmsg_nonblock)
						rescue ::IO::WaitReadable
							Celluloid::IO.wait_readable(@socket)
							retry
						end
					end
				end
			\end{lstlisting}
		\end{figure}
		
	\subsection{Concurrency and Performance}

		For maximizing the packet handling throughput, it is important how
		concurrency is accomplished on packet receiving. The main code
		regarding concurrent handling of network packets resides in
		\href{https://github.com/nning/david/blob/0.4.3/lib/david/server.rb}{\texttt{lib/david/server.rb}}.
		The event loop (line 41 ff. of that file) is also shown in
		\autoref{lst:david:eventloop}. The Rack handler starts the basic server
		actors \texttt{Server}, \texttt{Observe}, and
		\texttt{GarbageCollector}. Each actor runs in its own Thread and is
		supervised by \emph{Celluloid} (restarted on errors).  Afterwards the
		Rack handler code calls the \texttt{run} method on the \texttt{Server}
		actor blockingly (see
		\href{https://github.com/nning/david/blob/0.4.3/lib/rack/handler/david.rb#L15}{line
		15 of \texttt{lib/rack/handler/david.rb}}).
		
		The event loop receives incoming package data in a non-blocking way and
		calls the \texttt{dispatch} method for input handling (lines 3 and 6 of
		\autoref{lst:david:eventloop}). If the \texttt{IO} object is not
		readable, \texttt{Celluloid::IO} is used to select the next readable
		one (line 8). Benchmarks showed a significant performance increase
		through avoidance of calling \texttt{dispatch} asynchronously in a
		Fiber (by the \texttt{async} method provided by \emph{Celluloid}). It
		is important that blocking I/O operations are not used by the framework
		or application programmer, because the generation of the Rack
		application's response happens inside the event loop and it stops other
		incoming messages from being processed. Non blocking database adapters
		have to be used in the Rack application. Support for separate answers
		would especially be useful in this context.
			
	\subsection{Protocol Translations}
		
		The \ac{CoAP} methods are mapped directly to their \ac{HTTP}
		counterparts and passed to the Rack compatible framework via the
		\texttt{REQUEST\_METHOD} Rack environment entry. Messages with
		\ac{CoAP} codes that are undefined method codes (see section 12.1.1 of
		\cite{coap}) are dropped directly after parsing the message.

		Most direct mappings of \ac{HTTP} response codes and headers to
		\ac{CoAP} response codes and options are performed as specified in
		\autoref{cha:design:protocol} through the methods defined in
		\href{https://github.com/nning/david/blob/0.4.3/lib/david/server/mapping.rb}{\texttt{lib/david/server/mapping.rb}}.
		The methods are included into the \texttt{David::Server} class and used
		especially in its \texttt{respond} method defined in
		\href{https://github.com/nning/david/blob/0.4.3/lib/david/server/respond.rb}{\texttt{lib/david/server/respond.rb}}.
		The \ac{HTTP} response codes returned in Rack responses are mapped
		according to the procedures described in \autoref{cha:design:protocol}.
		In the Ruby code the map is represented through the constant
		\href{https://github.com/nning/david/blob/0.4.3/lib/david/server/mapping.rb#L6}{\texttt{HTTP\_TO\_COAP\_CODES}}.
		With the Rack option \emph{MinimalMapping}, the default mapping of
		\ac{HTTP} to \ac{CoAP} response codes can be deactivated in case
		application programmers want to explicitly return a certain code but
		can not use a Float (see
		\autoref{cha:evaluation:interoperability:rack}). Currently the
		implementation does not map \ac{HTTP} return code \textbf{and} methods
		to \ac{CoAP} codes. Also the translation of \ac{HTTP} redirects to
		\ac{JSON} redirect descriptions is not implemented. Both were queued up
		because of the time constraints of this work.
		
		The \texttt{Mapping} module also contains methods to convert certain
		headers from \ac{HTTP} to \ac{CoAP} or the other way round.
		\texttt{etag\_to\_coap} converts the \ac{HTTP} ETag header value string
		(containing a hash of the body) by interpreting its first bytes
		(defaulting to four) as an integer. \texttt{location\_to\_coap} splits
		the Location header string value into \ac{CoAP} Location-Path segments.
		\texttt{max\_age\_to\_coap} returns the Cache-Control header's max-age
		value as an integer. The method \texttt{accept\_to\_http} is used to
		translate the Accept header from \ac{CoAP} to \ac{HTTP} and is
		therefore enabling Content Negotiation (as specified in
		\autoref{cha:design:protocol:content}). The options Content-Format,
		Max-Age, Location-Path, and Size2 are mapped without individual methods
		of the \texttt{Mapping} module. The other direct header mappings
		drafted in \autoref{table:mapping:headers} (Proxy-Uri, Proxy-Scheme,
		Location-Query, If-Match, If-None-Match, and Size1) are not
		implemented, yet.

		Content transcoding of \ac{CBOR} to \ac{JSON} and vice versa is
		activated if the Rack environment option \texttt{CBOR} or the Rails
		option \texttt{config.coap.cbor} are set to \texttt{true}. It is
		deactivated by default. If the transcoding is activated and an incoming
		request has the content format \emph{application/cbor} (60), the
		message body is parsed by the Ruby cbor gem \cite{cbor-gem} as
		\ac{CBOR} into a Ruby object (see
		\href{https://github.com/nning/david/blob/0.4.3/lib/david/server/respond.rb#L31}{line
		31 of \texttt{lib/david/server/respond.rb}} and
		\href{https://github.com/nning/david/blob/0.4.3/lib/david/server/mapping.rb#L41}{line
		41 of \texttt{lib/david/server/mapping.rb}}). This object is saved in
		the Rack environment (as value of the key \texttt{coap.cbor}) and then
		converted to \ac{JSON}, wrapped in a StringIO instance and also handed
		down the Rack stack as Rack environment option \texttt{rack.input}. The
		Rack environment option \texttt{CONTENT\_TYPE} is set to
		\emph{application/json} if \ac{CBOR} parsing and \ac{JSON}
		serialization did not throw any exceptions. The
		\texttt{CONTENT\_LENGTH} is updated according to the length of the
		resource represented as \ac{JSON}. Rails integrates the deserialized
		object into the params Hash which is accessable in controller code.
		After a Rack response is obtained from the application, the response
		body is transcoded to \ac{CBOR} if its content type is
		\emph{application/json}. The resource representation is parsed as
		\ac{JSON} into a Ruby object and then serialized into \ac{CBOR} (see
		\href{https://github.com/nning/david/blob/0.4.3/lib/david/server/respond.rb#L50}{line
		50 of \texttt{lib/david/server/respond.rb}} and
		\href{https://github.com/nning/david/blob/0.4.3/lib/david/server/mapping.rb#L38}{line
		38 of \texttt{lib/david/server/mapping.rb}}).

	\subsection{Architecture}

		We made especially use of \emph{Celluloid} actors and the inclusion of
		modules into classes for modularization.
		\autoref{fig:implementation:object-relations-1} and
		\ref{fig:implementation:object-relations-2} show the object relations.
		The former concentrates on the actors mainly involved in request
		handling and response processing, whereas the latter shows Rack
		middleware and \ac{Rails} extensions. In both graphs names are suffixed
		with a letter in square brackets that indicates the type. \emph{C}
		stands for a Ruby class, \emph{M} for a model, and \emph{A} for a
		\emph{Celluloid} actor.
		
		% TODO	Configure rotation angle.
		\begin{figure}
			\centering
			\includegraphics[width=\textheight,angle=270]{images/object-relations-1.pdf}
			%\includegraphics[width=\textheight,angle=90]{images/object-relations-1.pdf}
			\caption{Object Relations (Server)}
			\label{fig:implementation:object-relations-1}
		\end{figure}

		\begin{figure}
			\centering
			\includegraphics[width=\textwidth]{images/object-relations-2.pdf}
			\caption{Object Relations (Middleware)}
			\label{fig:implementation:object-relations-2}
		\end{figure}

\section{Resource Discovery}
\label{cha:implementation:discovery}

	The provision of the \texttt{.well-known/core} interface for resource
	discovery on an application implemented in \ac{Rails} is solved through a
	Rack middleware that is also a \emph{Celluloid} actor. The middleware code
	mainly resides in
	\href{https://github.com/nning/david/blob/0.4.3/lib/david/resource_discovery.rb}{\texttt{lib/david/resource\_discovery.rb}}.
	Currently, it is \ac{Rails} specific, so applications written with other
	Rack supporting frameworks would have to implement Resource Discovery by
	their own. On invocation of \texttt{call}, the actor responds with a cached
	list of resources in the CoRE Link Format \cite{link} if the request is a
	GET on \texttt{.well-known/core}. The responded resources are automatically
	detected via \ac{Rails} routes and can be extended with link attributes by
	annotations in the controllers as depicted in
	\autoref{lst:discovery:annotation} and \autoref{img:discovery:annotation}.
	The annotation and registration of attributes at the Resource Discovery
	actor is accomplished by a class method named \texttt{discovery} defined on
	\texttt{ActiveController::Base} in
	\href{https://github.com/nning/david/blob/0.4.3/lib/david/rails/action_controller/base.rb}{\texttt{lib/david/rails/action\_controller/base.rb}}.
	Due to \ac{Rails} lazy loading controller classes in the \emph{development}
	environment, the \texttt{discoverable} class method is not called until the
	first time a route mapped to an action of that controller is requested. To
	overcome this limitation, link attributes for discovery would have to be
	specified together with the application routing.

	Especially for Service and Resource Discovery, the basics of
	\cite{coap-group} have been implemented in both David and the coap gem.
	The initialization of the socket is packaged in the \texttt{Multicast}
	module (see
	\href{https://github.com/nning/david/blob/0.4.3/lib/david/server/multicast.rb}{\texttt{lib/david/server/multicast.rb}}).
	The multicast groups \texttt{ff02::1}, \texttt{ff02::fd},
	\texttt{ff05::fd}, and \texttt{224.0.1.187} are joined by default (also see
	section 12.8 of \cite{coap}). Some workarounds had to be implemented
	specific to OS X. The OS X kernel does not support joining a \acs{IPv6}
	multicast group on the default interface if the interface index is set to 0
	(despite both the documentation and POSIX specification states otherwise),
	for example\footnote{\urlOsxMcast}. Utility methods for conversion of a
	network interface name to index and to test if an interface is activated
	have been integrated into the coap gem as extensions of the Ruby standard
	library \texttt{Socket} class (see
	\href{https://github.com/nning/coap/blob/0.1.1/lib/core/core_ext/socket.rb}{\texttt{lib/core/core\_ext/socket.rb}}).

	Other actors like \texttt{Server}, \texttt{Observe}, or
	\texttt{GarbageCollector} are instantiated and supervised named by calling
	\texttt{supervise\_as} instead of \texttt{new} (in
	\href{https://github.com/nning/david/blob/0.4.3/lib/rack/handler/david.rb#L7}{lines
	7-12 of \texttt{lib/rack/handler/david.rb}}). However, when a
	middleware is inserted into the stack with a Railtie, not an instance of
	the class but the class itself is passed. This way a discovery actor could
	not be supervised by \emph{Celluloid} and communicated with by name.
	Therefore, it has to register \enquote{manually} on initialization. This
	unfortunately does not mean it is also supervised. In this case, the
	solution is to proxy the actual Resource Discovery actor with a class
	behaving like an ordinary Rack middleware (as seen in
	\href{https://github.com/nning/david/blob/0.4.3/lib/david/resource_discovery_proxy.rb}{\texttt{lib/david/resource\_discovery\_proxy.rb}}).
	On initialization of this class, the Resource Discovery actor is
	instantiated and supervised named. When the \texttt{call} method is
	invoked, it is passed on to the actor.

\section{\acf{RD}}
\label{cha:implementation:core-rd}

	The \acl{RD} component is realized as a \ac{Rails} application
	(downloadable from GitHub\footnote{\urlCoreRd}) in order to further test
	\emph{David} and other implemented components regarding \ac{CoAP}
	application development. The Rack middleware for Resource Discovery shipped
	with David is not used in this case, because the \ac{Rails} application
	should provide a custom logic behind the \texttt{.well-known/core} resource
	and the middleware is tightly integrated with the routing and controllers.
	By default the \ac{CoAP} server sets the \texttt{Accept} header to
	\texttt{application/json} if it is not specified by the client. In the
	\ac{RD} application, the default content format is configured to
	\texttt{application/link-format}. Currently only the \emph{\ac{RD} Function
	Set} (see section 5 of \cite{rd}) is implemented completely. The
	\emph{\ac{RD} Lookup Function Set} (see section 7) is only implemented in a
	way that allows for the lookup of resources (\emph{res} lookup type). We
	did not implement the \emph{Group Function Set} (see section 6) or the
	lookup types related to that set.

	There are Active Record models for Resource Registrations, Typed Links, and
	their Target Attributes (see
	\href{https://github.com/nning/core-rd/blob/0.1.0/app/models}{the
	\texttt{app/models} directory in the \texttt{core-rd} repository}).
	Controllers exist per function set and for the \texttt{.well-known/core}
	ressource (see
	\href{https://github.com/nning/core-rd/tree/0.1.0/app/controllers}{\texttt{app/controllers}}).
	We developed the application utilizing \ac{TDD} with RSpec.
