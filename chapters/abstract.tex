\chapter*{Abstract}

% TODO	More to the point; focus on results!

\section*{}

We present a \acf{CoAP} server with a Rack interface to enable application
development for the \acl{IoT} (or Wireless Embedded Internet) using frameworks
such as \acl{Rails}. Those frameworks avoid the need for reinvention of the
wheel, and simplify the use of \ac{TDD} and other agile software development
methods. They are especially beneficial on less constrained devices such as
infrastructure devices or application servers. Our solution supports
development of applications almost without paradigm change compared to
\acs{HTTP} and provides performant handling of numerous concurrent clients. The
server translates transparently between the protocols and also supports
specifics of \ac{CoAP} such as service and resource discovery, block-wise
transfers and observing resources. It also offers the possibility of
transparent transcoding between \acs{JSON} and \acs{CBOR} payloads. The
\acl{RD} draft was implemented by us as a \acs{Rails} application running on
our server software.

\section*{}

Wir stellen einen \acf{CoAP} Server mit einem Rack Interface vor, der
Anwendungsentwicklung für das Internet der Dinge (bzw. das Wireless Embedded
Internet) mit Frameworks wie \acl{Rails} ermöglicht. Solche Framworks
verhindern die Notwendigkeits, das Rad neu zu erfinden und vereinfachen die
Anwendung testgetriebener Entwicklung (\acs{TDD}) und anderer agiler Methoden
der Softwareentwicklung. Sie sind vor allem auf weniger eingeschränkten Geräten
wie Infrastrukturgeräten und Applikationsservern vorteilhaft. Unsere Lösung
ermöglicht Applikationsentwicklung nahezu ohne Pradigmenwechsel verglichen mit
\ac{HTTP} und bietet performante Handhabung von zahlreichen nebenläufigen
Clients. Der Server übersetzt transparent zwischen den Protokollen und
unterstützt ebenso Besonderheiten von \acs{CoAP} wie Service und Resource
Discovery, Block-wise Transfers und Observing Resources. Er bietet zusätzlich
die Möglichkeit, transparent zwischen \acs{JSON} und \acs{CBOR} Payloads zu
transkodieren. Den \acl{RD} Entwurf haben wir als \acs{Rails} Anwendung
implementiert, die auf unserer Server Software läuft.
