\chapter{Background}
\label{cha:background}

For the design of different aspects of the software created during this work,
some background on the server's concurrency architecture, the Rack interface,
and the \ac{CoAP} implementation will be given in this chapter. We will weigh
the advantages and disadvantages of different concurrency models in
\autoref{cha:background:concurrency}. The Rack interface is introduced in
\autoref{cha:background:rack}. For the \ac{CoAP} implementation, we analyze
different existing projects to learn about implementation techniques in
\autoref{cha:background:coap}. Some background will be given about the Observe
extension \cite{observe} and the \ac{CBOR} \cite{cbor}.

\section{Concurrency and Celluloid}
\label{cha:background:concurrency}

	% TODO	Explain non-blocking
	% TODO	Celluloid concurrent data structures?

	Classicly the concurrent handling of network messages is based on
	processes, threads, or a combination of these. Both have some performance
	drawbacks through overhead regarding process state and context switching.
	In general, the overhead of threads is a little lower than with processes.
	
	Using threads in Ruby (\ac{MRI}) even has some extra performance issues:
	Although Ruby threads are mapped to \ac{OS} level threads, a \ac{GIL}
	prevents scaling to more than one \ac{CPU} per process and therefore
	also prohibits maximum concurrency. Threads also provide some
	implementation challenges such as avoiding deadlocks. Process-based
	concurrency tends to an even higher memory consumption, if there is a
	necessity to share state between processes. Kovatsch and Erb give a
	detailed overview of web server architectures and their advantages and
	disadvantages for scalability and performance in the publications
	\citetitle{scalable-iot} \cite{scalable-iot} (section 4.1.2) and
	\citetitle{scalable-web} \cite{scalable-web}, respectively.

	Event-driven architecture attempts to reduce the overhead caused through
	process or thread based concurrent handling of network messages. The
	Reactor pattern \cite{reactor-1} \cite{reactor-2} provides a pattern for
	implementation of event-driven architecture. Popular examples of Reactor
	pattern implementations in the Ruby ecosystem are \emph{eventmachine} and
	\emph{Celluloid::IO}. \emph{Celluloid::IO} provides drop-in replacements
	for the socket classes from the Ruby standard library that use the reactor
	pattern on strictly non-blocking receive methods. It is built on top of
	\emph{Celluloid} -- a library that provides an \enquote{actor-based
	concurrent object framework} \cite{celluloid}. So \emph{Celluloid} does not
	only provide a library for evented I/O but also a framework based on
	threads for object concurrency in general. \emph{Celluloid} uses fibers for
	asynchronous message calls between threads. Fibers are supported since Ruby
	1.9 as a manually scheduled means of concurrency that needs less memory and
	context changes than threads. The \emph{Celluloid} README
	\cite{celluloid-readme} states that \enquote{Each actor is a concurrent
	object running in its own thread, and every method invocation is wrapped in
	a fiber that can be suspended whenever it calls out to other actors, and
	resumed when the response is available.} Celluloid is modeled after the
	ACTOR Formalism defined by Hewitt et al.\ in 1973. Actors integrate well
	with clean, synchronous object oriented design, because no callbacks are
	used for asynchronous operations. They can be supervised and automatically
	be restarted on crashes through unhandled exceptions for example. Linking
	one actor to another ensures the former also crashes or being notified when
	the latter fails. \emph{eventmachine} only provides evented I/O and
	therefore object concurrency (application threads, messaging, etc.) would
	have to be approached manually when it is used. Another argument against
	\emph{eventmachine} is its flawed \acs{IPv6} support \footnote{\urlEMSix}.

	The server process to be implemented can consist of several concurrent
	actors for different tasks such as handling new requests or managing
	notifications to clients observing resources. The actors handling incoming
	packages can utilize reactors provided by \emph{Celluloid::IO}
	\cite{celluloid-io} to handle concurrent network communications. 

\section{Rack}
\label{cha:background:rack}

	% TODO	One contextual sentence..?

	The basic ideas of Rack are as follows. An application is a Ruby object
	that responds to a method named \texttt{call}. This method is invoked for
	example by a server when an application response is demanded. The only
	argument an invocation of \texttt{call} includes is a
	\texttt{Hash}\footnote{\urlRubyHash} representing the environment of the
	request. This includes information such as the \ac{HTTP} method (see
	section 4 of \cite{http-1}), the path, query string and \ac{HTTP}
	headers (see section 5 of \cite{http-1}) of the request. The Rack
	environment was inspired by the \ac{CGI}, an interface for web servers to
	retrieve \ac{HTTP} responses from external programs. It is also extended
	by Rack specific options such as the Rack interface version. An invocation
	of \texttt{call} has to return an \texttt{Array}\footnote{\urlRubyArray}
	containing the \ac{HTTP} status code (see section 6 of \cite{http-1}), a
	\texttt{Hash} of \ac{HTTP} headers, and the response payload\footnote{Which
	has to respond to \texttt{each}, take a block and yield chunked data as
	\texttt{Strings}}. An example for a particularly minimal Rack application
	would be the Ruby code shown in \autoref{lst:rack:basic}; normally headers
	such as \emph{Content-Type} and \emph{Content-Length} would be included
	with their respective values.

	\begin{figure}[H]
		\begin{lstlisting}[gobble=6,caption={Minimal Rack application},label=lst:rack:basic]
			# A Proc object getting env as argument returning an Array.
			->(env) { [200, {}, ['Hello World']] }
		\end{lstlisting}
	\end{figure}

	Rack offers the possibility to wrap the invocation of the \texttt{call}
	method of an application through an intermediary called \emph{middleware}.
	A piece of middleware basically works like an application, also providing a
	\texttt{call} method with the environment as argument. The Rack application
	is passed upon initialization of the middleware class. On \texttt{call}
	invocations, the middleware is able to change the environment or the
	application response according to its function. As an example, an
	authentication mechanism would be possible to implement as a Rack
	middleware that returns a forbidden status code unless the request
	environment contains valid credentials in the query string. A popular Rack
	middleware example included in \ac{Rails} by default and also interesting
	in the context of this work is \emph{Rack::ETag} which adds entity tag
	headers to responses (see section 2.3 of \cite{http-2}). Rack middleware is
	usually independent from the framework or application, so it can be used
	with different frameworks supporting Rack.

	Instance variable access in Rack middleware can lead to concurrency
	problems when conducted in a multi-threaded system, because every
	middleware class is only instantiated once\footnote{\urlRackDupCall}. See
	\autoref{lst:rack:dupcall} for a possible but rather expensive workaround.

	\begin{figure}[H]
		\begin{lstlisting}[gobble=6,caption={Rack dup.\_call workaround},label=lst:rack:dupcall]
			class Middleware
				def call(env)
					dup._call(env)  # Duplicate instance and invoke _call.
				end

				def _call(env)      # Actual middleware logic.
					@env = env      # Safe now!
					...
				end
			end
		\end{lstlisting}
	\end{figure}

	There are numerous different Rack frameworks, some of them with very
	distinctive features. We compiled a list of frameworks and their download
	locations in \autoref{table:background:rack} for an overview. We will refer
	to some of them later in other contexts.

	\begin{table}[H]
		\begin{center}
			\begin{tabular}{l|l}
				Framework			& Website \\
				\hline
				Grape				& \url{http://intridea.github.io/grape} \\
				Camping				& \url{http://camping.io} \\
				Cramp				& \url{https://cramp.in} \\
				Cuba				& \url{https://cuba.is} \\
				Brooklyn			& \url{https://github.com/luislavena/brooklyn} \\
				Hobbit				& \url{https://github.com/patriciomacadden/hobbit} \\
				Nancy				& \url{http://guilleiguaran.github.io/nancy} \\
				NYNY				& \url{http://alisnic.github.io/nyny} \\
				Ramaze				& \url{http://ramaze.net} \\
				\acs{REST}Rack		& \url{http://restrack.me} \\
				Roda				& \url{http://roda.jeremyevans.net} \\
				Scorched			& \url{http://scorchedrb.com} \\
				Sinatra				& \url{http://www.sinatrarb.com} \\
				\acs{Rails}::API	& \url{https://github.com/rails-api/rails-api} \\
				\acl{Rails}			& \url{http://rubyonrails.org} \\
			\end{tabular}
		\end{center}
		\caption{Rack frameworks}
		\label{table:background:rack}
	\end{table}

	As the Rack interface is designed for \ac{HTTP} requests and responses,
	some adaption has to be performed when using it with \ac{CoAP}. On
	requests, \ac{CoAP} options have to be translated to \ac{HTTP} headers to
	be included in the Rack environment. For responses -- besides this
	options/headers mapping in the opposite direction -- also \ac{HTTP}
	response codes have to be translated. Other functionality such as observing
	resources or resource discovery has to be adapted, too.

\section{\acs{CoAP}}
\label{cha:background:coap}

	The \acl{CoAP} is standardized as \cite{coap} by the \ac{IETF}. It
	incorporates successful design elements also used in \ac{HTTP}. In contrast
	to \ac{HTTP} it is binary, based on \ac{UDP} by default and not bound to
	synchronous messaging or long-lasting connections. Its default ports are
	\texttt{5683/udp} for unencrypted messages and \texttt{5684/udp} when
	secured with \ac{DTLS}. Messages can be transferred in a reliable
	(\emph{Confirmable}) or unreliable way (\emph{Unconfirmable}). Reliably
	transferred messages are responded to with an acknowledge message by the
	receiver. If a confirmable message is not responded to, it gets
	retransmitted in exponentially increasing intervals. A response can either
	occur directly (piggy-backed) or separately if generating a response would
	take longer than the retransmission timeout. In this case an empty message
	is sent first that confirms the request. When the actual response is ready,
	it gets sent as a new confirmable transmission. Unconfirmable messages are
	not retransmitted and responses are always separated.
	
	\ac{CoAP} specifies the methods \texttt{GET}, \texttt{POST}, \texttt{PUT},
	and \texttt{DELETE} for retrieval, creation, update, and deletion of
	resources. Responses carry status codes similar to \ac{HTTP} but less
	complex. The headers of \ac{CoAP} messages can contain numerous options
	comparable to \ac{HTTP} header fields. They are used for functionality such
	as Cache Validation and Conditional Requests (\texttt{ETag},
	\texttt{If-Match}), or Content Negotiation (\texttt{Accept},
	\texttt{Content-Format}).

	As \ac{CoAP} is not necessarily reliable and synchronous, it is possible to
	use multicast messaging to address a group of hosts with a single
	message\footnote{Although messages effectively are broadcasted on some
	physical layers such as \cite{802-15-4}}. This can also be used for
	discovery of other reachable \ac{CoAP} nodes and is called \emph{Service
	Discovery} (see section 7.1 of \cite{coap}). A special resource with the
	path \texttt{.well-known/core} is specified to return information about the
	available resources of a node. This procedure is referred to as
	\emph{Resource Discovery} (see section 7.2). As the maximum payload size of
	a \ac{CoAP} message is around 1 KiB, larger resource representations have
	to be transmitted in several messages as it is defined in an extension for
	\emph{Block-wise Transfers} \cite{block}. Another extension named
	\emph{Observing Resources} specifies a way to register for and subsequently
	receive updates about the state of a resource \cite{observe}.

	\subsection{Existing Implementations}
	\label{cha:background:coap:implementations}

		In the context of the research questions, it is interesting whether an
		implementation supports block-wise transfers \cite{block}, observe
		\cite{observe} and (multicast) group communications \cite{coap-group}.
		\autoref{table:background:coap:libraries} lists Open Source
		implementations obtained from the \ac{CoAP} article in the Wikipedia
		\cite{coap-list-1} and the coap.technology website \cite{coap-list-2}
		that support server mode, block-wise transfers and observe. The
		coap.technology website lists a pure Ruby implementation developed in
		the student project \emph{GOBI} of the communication networks research
		group (AG Rechnernetze\footnote{\urlAgrn}) at the Universität Bremen
		based on a \ac{CoAP} message parser by Carsten Bormann. Information
		about compatibility with Ruby and group communications support is added
		to the compiled list. A library is seen as compatible if it is written
		in Ruby itself, C/C++ or Java. Whether multicast communications are
		supported has been determined by analyzing the respective source code
		and searching for group addresses specified by section 12.8 of
		\cite{coap-group}. There are libraries like \emph{libcoap} that are
		multicast compatible but do not listen on \ac{CoAP} group addresses by
		default or leave it to the programmer to do so. \emph{CoAPthon} and
		\emph{SMCP} are the only implementations listening on the \ac{CoAP}
		group addresses by default\footnote{We also tested incompatible
		implementations.}.

		\begin{table}[H]
			\begin{center}
				\begin{tabular}{l|l|l}
					Library				& Compatibility	& Multicast (\cite{coap-group}) \\
					\hline
					Californium			& \y			& \n \\
					CoAPthon			& \n			& \y (\acs{IPv4}) \\
					COAP.NET			& \n			& \n \\
					Erbium for Contiki	& \n			& \n \\
					jcoap				& \y			& \n \\
					libcoap				& \y			& \n \\
					node-coap			& \n			& \n \\
					Ruby coap (GOBI)	& \y			& \n \\
					SMCP				& \y			& \y (\acs{IPv6}) \\
					txThings			& \n			& \n \\
				\end{tabular}
			\end{center}
			\caption{CoAP Libraries -- Possible compatiblity with Ruby and
				multicast support}
			\label{table:background:coap:libraries}
		\end{table}

		% TODO	Some protocol implementation details of other libraries

		In the following paragraphs, we briefly examine the source code of
		\emph{Californium} and \emph{libcoap} with regards to their
		architecture, implementations of the \ac{CoAP} Message Deduplication
		(see section 4.5 of \cite{coap}) and concurrency approaches.
		\autoref{table:background:coap:libraries:versions} lists the examined
		versions.

		\begin{table}[H]
			\begin{center}
				\begin{tabular}{l|l}
					Library				& Version \\
					\hline
					Californium			& \href{https://github.com/eclipse/californium/tree/1.0.0-M3}{1.0.0-M3} (Core) \\
					& \href{https://github.com/eclipse/californium.element-connector/tree/1.0.0-M3}{1.0.0-M3} (Connector) \\
					libcoap				& \href{http://sourceforge.net/p/libcoap/code/ci/d48ab449fd05801e574e4966023589ed7dac500b/tree}{\texttt{d48ab44}} \\
%					Ruby coap			& \href{https://github.com/SmallLars/coap/tree/1cd124427b57e84543f3ee6ff2d9148e8499393a}{\texttt{1cd1244}} \\
%					SMCP				& \href{https://github.com/darconeous/smcp/tree/08e822e86a39e74ce07e0f675980723c70c508cf}{\texttt{08e822e}} \\
				\end{tabular}
			\end{center}
			\caption{CoAP Libraries -- Examined versions}
			\label{table:background:coap:libraries:versions}
		\end{table}

		\subsubsection{Architecture}

			% TODO	Californium Exchange class?

			Some architectural inspirations will be mentioned but we will not
			analyze the overall architecture of every library.

			\emph{Californium} uses an approach similar to the design of Rack
			to implement different functional protocol \enquote{layers} (see
			\href{https://github.com/eclipse/californium/blob/1.0.0-M3/californium-core/src/main/java/org/eclipse/californium/core/network/CoAPEndpoint.java#L83}{line
			83 of \texttt{CoAPEndpoint} class (Core)}). Messages are passed
			through a stack of classes each handling a certain transmission
			layer task. A \texttt{ReliabilityLayer}, for example, retransmits
			messages if there was no response; a \texttt{BlockwiseLayer}
			handles block-wise transfers.

		\subsubsection{Message Deduplication}

			% TODO	Kontext auf Papier unklar.

			\emph{Californium} makes use of the Java class
			\texttt{ConcurrentHashMap} as a hash table for managing data about
			ongoing transmissions (see
			\href{https://github.com/eclipse/californium/blob/1.0.0-M3/californium-core/src/main/java/org/eclipse/californium/core/network/Matcher.java#L58}{line
			58 to 61 of \texttt{Matcher} class (Core)}). Three hash tables are
			used with different keys (message ID, token, and URI). The Java
			documentation describes \texttt{ConcurrentHashMap} as a
			\enquote{hash table supporting full concurrency of retrievals and
			adjustable expected concurrency for updates}
			\cite{java-concurrenthashmap}.

		\subsubsection{Concurrency}
		\label{cha:background:coap:concurrency}

			\emph{Californium} maintains a pool of threads which send or
			receive messages (see
			\href{https://github.com/eclipse/californium.element-connector/blob/1.0.0-M3/src/main/java/org/eclipse/californium/elements/UDPConnector.java}{\texttt{UDPConnector}
			class (Connector)}). Other parts of the code asynchronously
			schedule messages to be sent (see for example
			\href{https://github.com/eclipse/californium/blob/1.0.0-M3/californium-core/src/main/java/org/eclipse/californium/core/network/stack/ReliabilityLayer.java#L178}{line
			178 of \texttt{ReliabilityLayer} class (Core)}).
			\emph{Californiums} use of \texttt{ConcurrentHashMap} ensures safe
			and performant concurrent access to data shared by several threads.

			% TODO	Kontext auf Papier unklar.

			\emph{libcoap} itself does not implement a server ready to use but
			provides the \ac{API} for programmers to base their own server
			implementations on. The source code contains an example server that
			is realized as a single threaded event loop (see
			\href{http://sourceforge.net/p/libcoap/code/ci/d48ab449fd05801e574e4966023589ed7dac500b/tree/examples/server.c#l414}{line
			414 of \texttt{examples/server.c}}).

	\subsection{Server-sent Updates}
	\label{cha:background:coap:observe}

		% TODO	More RFC6202?
		% TODO	More SSE?

		In the \ac{HTTP} world, some protocols are designed to overcome the
		strictly synchronous request model of HTTP/1.1 (see \cite{http-0} ff.)
		and are therefore in some extent comparable to the \ac{CoAP} Observe
		extension \cite{observe}. Their properties differ in distinct ways from
		Observe, which is not keeping connections open, is unidirectional and
		does not strictly require the server pushing updates only directly
		after requests of the client. \emph{Long Polling} and \emph{Streaming}
		are two mechanisms common to achieve server-initiated communication
		with \ac{HTTP}. \cite{http-bidir} summarizes known issues and lists
		best practices regarding to the use of these mechanisms. Popular
		choices for this purpose in the traditional web are
		Comet\footnote{\urlComet} and WebSockets \cite{websocket}. Both use
		long opened connections, which puts extra load on the servers. Comet
		just keeps a \ac{HTTP} GET request opened, which is used by the server
		to push updates gradually. WebSocket specifies bi-directional
		out-of-band (non-\ac{HTTP}) communications. Solutions more similar to
		Observe are \ac{HTTP}/2 Server Push (see section 8.2 of \cite{http2})
		and \acs{HTML5} Server-Sent Events \cite{sse}, which both are
		unidirectional but also employ long-running connections. \ac{HTTP}/2
		Server Push is not truly asynchronous, because the push mechanism is
		only used to multiplex documents anticipating a direct request.
		\acs{HTML5} Server-Sent Events are most similar to Observe from an
		architectural perspective. The implementation of the mentioned
		protocols in connection with Rack and \ac{Rails} can give valuable
		input for the design of the Observe integration in the context of this
		work. Therefore some existing Ruby gems that implement the mentioned
		techniques can be a source of inspiration. We further analyze that
		topic in \autoref{cha:design:protocol:observe}.

	\subsection{Payload Formats}
	\label{cha:background:coap:payload}

		There are a number of different formats for resource representation
		(serialization) that are more or less optimized on low resource usage
		(when transmitted or parsed). In the world of less constrained web
		applications especially \ac{JSON} and \ac{XML} are common as resource
		representations for \acp{API}. Both are convertible into less resource
		demanding binary serialization formats. For \ac{JSON} there are for
		example \ac{BSON}, \ac{UBJSON}, MessagePack and \ac{CBOR}. \ac{XML} can
		be converted into \ac{EXI} to save resources. We will concentrate on
		\ac{JSON}, because it currently is more widespread than \ac{XML} in the
		context of modern web application development. Especially \ac{CBOR} is
		considered as a binary serialization format due to its closeness to
		\ac{JSON}, the compactness (of code and data), and the possibility to
		be used without a schema description (see section 1.1 of \cite{cbor}).

		% As with \ac{CoAP}, it is also standardized by the \ac{CoRE} \ac{WG}
		% of the \ac{IETF}.
