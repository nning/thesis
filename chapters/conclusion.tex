\chapter{Conclusion}
\label{cha:conclusion}

\section{Recapitulation of Results}

	The following subsection lists our accomplishments and the deficiencies of
	our implementation for each research question.

	\subsection*{How can \ac{CoAP} be implemented?}

		We iteratively designed and developed the protocol handling for a
		\ac{CoAP} server. Opposed to our previous estimations, we did not
		depend on larger parts of a shared library for the \ac{CoAP}
		transmission layer but only for message parsing and assembly. The
		implementation makes use of Message Deduplication to lower the waste of
		server resources through retransmitted confirmable messages. Resource
		representations are requestable utilizing Block-wise transfers. Our
		client interoperability evaluation showed a very good protocol
		interoperability, although we would favor a more polished \ac{CoAP}
		transmission implementation that supports separate responses,
		confirmable messages initiated by the server, and incoming Block-wise
		Transfers. A structure comparable to Rack's middleware stack for
		different protocol layers would probably have improved code modularity
		and quality.

	\subsection*{How to handle a huge amount of \ac{IoT} devices?}

		Message handling was implemented utilizing a single threaded, non
		blocking event loop as a model providing a good balance between
		performance and Ruby \ac{VM} interoperability. In the performance
		evaluations, our realization of this model yielded a message throughput
		almost three times higher than previously anticipated. Also, our server
		has almost twice the throughput as Ruby \ac{HTTP} servers running on
		\ac{MRI}. However, further performance optimizations that make closer
		use of the different Ruby \ac{VM} characteristics would be desirable.

	\subsection*{Does the Rack interface comply with the characteristics of
		\ac{CoAP}?}

		We showed with our implementation of the Rack interface that it is
		basically also working in a sufficient way in conjunction with
		\ac{CoAP}. Some functionality was implemented as Rack middleware by us
		(\texttt{David::ShowExceptions}, \texttt{David::ResourceDiscovery}).
		The framework interoperability tests showed that the server
		implementation of the Rack interface is compatible to numerous
		different Rack based web frameworks. The only major restriction of the
		Rack interface is the problematic integration of asynchronous
		responses. The performance of our implementation in average cases could
		probably be improved by integrating a caching of Rack application
		responses.

	\subsection*{How can a later integration of transport encryption with
		\ac{DTLS} \cite{dtls} be supported?}

		We provide a basic interface for a \ac{DTLS} integration and means to
		pass details of the encrypted connection to the Rack middleware stack
		and the application.

	\subsection*{How can headers be translated between \ac{HTTP} and \ac{CoAP}?}

		According to our design, the server transparently translates \ac{CoAP}
		requests into Rack environments and Rack application responses into
		\ac{CoAP} responses. The framework and client interoperability tests
		showed that this translation works well. Only the explicit returning of
		\ac{CoAP} response codes, while a status code mapping from \ac{HTTP} to
		\ac{CoAP} is activated, keeps to be somewhat cumbersome. We also did
		not implement all possible translations as Conditional Request Options
		for example (see section 5.10.8 of \cite{coap}).

	\subsection*{How can payload formats be translated between their suitable
		fields of application?}

		Our design and implementation provides an automatic and transparent
		transcoding between \ac{JSON} and \ac{CBOR} resource representations.
		However, it is only activatable for an controller or action of the
		application. We also would have liked a more extensive and generalized
		transcoding implementation as a Rack middleware.

	\subsection*{Is the \ac{CoAP} Resource Discovery implementable
		transparently for the Rack application?}

		We provide a solution for a transparent discovery of the resources of a
		\ac{Rails} application as a Rack middleware. In contrast to most other
		\ac{CoAP} implementations, we integrated multicast communications also
		supporting the \ac{CoAP} multicast groups. The attributes of resources
		like interface descriptions are configurable from the \ac{Rails}
		controllers. Unfortunately, we could not provide a Rack application and
		framework agnostic way of resource autodetection.

	\subsection*{Can a \acl{RD} be realized with the created software?}

		The \ac{RD} and Lookup Funtion Sets of the \ac{RD} draft \cite{rd} have
		been implemented as a \ac{Rails} application by us. The development
		process showed that our solution is compatible to the specifications of
		that draft and that \ac{TDD} with RSpec works for \ac{CoAP} application
		development with \ac{Rails}. However, the Group Function Set is not
		supported yet. The Lookup Function Set was only implemented for
		resource lookup.

	\subsection*{Is it possible to support Observe?}

		We implemented the Observe extension based on \emph{Celluloid} actors.
		The determination whether updates are necessary is solved utilizing
		entity tags. We also designed a system based on \ac{HTTP} Caching to
		minimize the overhead of update checks. Our interoperability
		evaluations showed that the implemented observe support works with
		different clients and frameworks. Some implementation details such as
		garbage collection of observe relationships, caching on update checks,
		and deregister through reset messages were not realized by us.

\section{Perspectives}

	Implementing solutions to the deficiencies mentioned in the preceding
	accomplishment summary is a next step for the further existence of the
	created software system as an Open Source project. Especially the polishing
	of the \ac{CoAP} transmission implementation, further performance
	optimizations, working \ac{DTLS}, and protocol translations in a more
	detailed way are important goals. For performance optimizations the
	examination of the architectures \acs{AMPED}, \acs{SEDA} and PIPELINED
	would be interesting (see section 4.1.2 of \cite{scalable-iot}). The
	transcoding functionality of the server should be decoupled, generalized to
	work with other content formats, and -- as a Rack middleware -- made
	independent of \ac{CoAP}. The possibility to activate transcodings for
	single controllers or actions would be a useful feature. As the
	autodetection of resources for the Resource Discovery currently only works
	with \ac{Rails}, compatiblity to other Rack compatible frameworks would be
	an improvement. As mentioned before, the \ac{RD} application and the
	observe implementation are also providing some possibilities to be further
	improved. We will further develop and maintain the open source projects
	started during this work and probably try to get other people involved.
	That includes the Ruby \ac{CoAP} library for which also numerous
	improvements are possible.

	Beyond the concrete implementation created during this work, there are some
	possible applications of parts of our design in the context of other
	\ac{CoAP} servers. We would like to see solutions for hosting \ac{IoT} web
	applications developed in Ruby comparable to nginx\footnote{\urlNginx} and
	Phusion Passenger\footnote{\urlPassenger} for example as well as \ac{CoAP}
	servers compatible to web frameworks written in programming languages
	optimized for concurrency such as elixir\footnote{\urlElixir} for example.
